\chapter{Introduzione}

Questa tesi presenta una modifica a Redis, un popolare database NoSQL, introducendo i filtri di
Bloom quale struttura dati aggiuntiva a quelle già presenti. In particolare, 
l'aggiunta è utile per arricchire il sempre crescente numero di strutture dati che Redis offre agli
utilizzatori come strumento di catalogazione dell'informazione.

Sono sempre stato interessato alle strutture dati probabilistiche fin da quando le ho scoperte; la
capacità di poter arbitrariamente disfarsi di una parte di informazione mantenendone un sottoinsieme
che, per quanto non sufficiente a ricostruire l'intero set di dati, è comunque sufficiente ad alcuni
scopi, è sempre stata per me affascinante, in quanto strettamente legata all'ottimizzazione e
all'efficienza degli algoritmi.

L'obiettivo di questo lavoro è stato quindi quello di approfondire un database NoSQL che ho avuto
modo di utilizzare, Redis, studiandone più da vicino il funzionamento e contemporaneamente
potenziandone le funzionalità. Redis implementa già ad oggi una struttura dati probabilistica
(HyperLogLog) e, avendo trovato riferimenti dell'interesse dell'autore ad espandere il progetto in
questo senso, ho ritenuto che fosse un percorso di ricerca adatto alle mie competenze e alla mia
passione. Ho scelto dunque di concentrarmi sui filtri di Bloom, una struttura dati utilizzata
per ottenere test di appartenenza probabilistici, una funzionalità attualmente assente da
Redis ma che è ampiamente richiesta nel panorama informatico odierno e dunque perfettamente
allineata alla continua evoluzione dei database, volta ad ampliare le possibilità offerte a
supporto di problemi implementativi di scala sempre più ampia.

La tesi è articolata in cinque capitoli: dopo questa introduzione, nel secondo capitolo viene
descritto il funzionamento e l'architettura di Redis, descrivendo il contesto scientifico e
industriale che ha portato alla sua nascita, gli scenari applicativi principali in cui viene
utilizzato, le strutture dati che offre, le funzionalità più avanzate e le caratteristiche di
persistenza. Nel terzo capitolo invece vengono descritti i filtri di Bloom, studiandone gli scenari
applicativi, le principali caratteristiche, gli algoritmi e analizzando nel dettaglio come scegliere
i parametri che li definiscono. Nel quarto capitolo viene presentato il lavoro svolto di modifica
al codice sorgente di Redis, sottolineando le principali sfide incontrate e i risultati raggiunti in
termini di funzionalità e performance. Nel quinto ed ultimo capitolo si commentano i risultati
ottenuti e si descrive una possibile evoluzione del lavoro.

Il lavoro di questa tesi è disponibile online ed è stato inviato agli autori di Redis per
proporlo come integrazione nella base di codice ufficiale.
