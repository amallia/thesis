\chapter{Aggiunta dei filtri di Bloom a Redis}

\section{Obiettivo}

L'obiettivo della modifica a Redis svolta per questa tesi è introdurre il supporto per i filtri di
Bloom. È stato effettuata prima una fase di progettazione dei nuovi comandi da aggiungere e di
conseguenza della variante esatta di filtro da implementare; è stata preparata una prima
implementazione completa di copertura tramite test automatizzati; infine, sono state effettuate
delle analisi sui parametri di configurazione di default, cercando di studiare il compromesso tra
utilizzo della memoria, tempo di esecuzione, e probabilità di falsi positivi.

\section{Progettazione dei nuovi comandi}

Nel progettare i nuovi comandi, abbiamo cercato di adeguarci il più possibile allo stile dei
comandi esistenti, cercando di
