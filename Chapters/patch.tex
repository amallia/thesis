\chapter{Aggiunta dei filtri di Bloom a Redis}

\section{Obiettivo}

L'obiettivo della modifica a Redis svolta per questa tesi è introdurre il supporto per i filtri di
Bloom. È stato effettuata prima una fase di progettazione dei nuovi comandi da aggiungere e di
conseguenza della variante esatta di filtro da implementare; è stata preparata una prima
implementazione completa di copertura tramite test automatizzati; infine, sono state effettuate
delle analisi sui parametri di configurazione di default, cercando di studiare il compromesso tra
utilizzo della memoria, tempo di esecuzione, e probabilità di falsi positivi.

\section{Scelta della variante di filtro}

Nella fase iniziale di design, abbiamo comunicato più volte con l'autore di Redis (Salvatore
Sanfilippo), cercando di stabilire insieme i requisiti che l'implementazione del filtro dovesse
avere perché la modifica corrispondesse agli standard di Redis.

In particolare, abbiamo identificato i seguenti requisiti:

\begin{itemize}
	\medskip
	\item \textbf{Dinamicità}: tutte le strutture dati di Redis sono dinamiche, cioè è possibile
	aggiungere elementi illimitati, con l'unico vincolo rappresentato dalla memoria a disposizione
	sul sistema. Nessuna struttura dati richiede un dimensionamento fisso.

	\item \textbf{Scalabilità}: l'utilizzo della memoria deve essere molto contenuto quando la
	struttura dati contiene pochi elementi, e crescere gradualmente all'aumentare degli elementi.
	Eventuali pre-allocazioni devono essere molto limitate. Per esempio, non sarebbe accettabile che
	il filtro possa occupare \SI{1}{\mega\byte} di RAM alla creazione.

	\item \textbf{Scarsa configurabilità}: Redis cerca di offrire strutture dati che siano
	preconfigurate con default accettabili per la maggior parte dei casi d'uso, e offre la minima
	configurabilità necessaria ad operare negli scenari applicativi previsti. In particolare,
	i filtri di Bloom hanno diversi parametri a disposizione ed è quindi richiesto delle scelte
	oculate.
\end{itemize}

Per soddisfare i requisiti, ci siamo orientati su implementare i filtri di Bloom
scalabili (\autoref{sec:bloomscalable}), i quali eliminano la necessità di determinare in anticipo il
dimensionamento della struttura (in termini di byte o di numero di elementi), e la cui crescita
esponenziale di occupazione della memoria consente una sostanziale scalabilità.

Abbiamo ritenuto essenziale esporre il parametro $P$ di probabilità di falso positivo poiché
strettamente legato al tipo di scenario applicativo, ma nascondendo all'utente tutti i parametri più
tecnici (fattore di intensificazione $r$, fattore di crescita $s$, tipologia di funzione di hash).
Anche per il parametro $P$ ci siamo orientati su offrire un default, rendendo più semplice per
l'utente iniziare l'utilizzo della struttura dati, anche in fase di sviluppo.

\section{Progettazione dei nuovi comandi}

Nel progettare i nuovi comandi, abbiamo cercato di adeguarci il più possibile allo stile dei
comandi esistenti, cercando di mantenere le proprietà evinte durante la fase di analisi.

Abbiamo scelto il prefisso \verb|BF| per la famiglia di comandi, indicando chiaramente la
struttura dati che viene manipolata.

Questi sono i comandi implementati:

\begin{itemize}
	\medskip
	\item \textbf{BFADD <key> [ERROR p] ELEMENTS <ele1> <ele2> ...}: aggiunge uno o più elementi al
	filtro di Bloom specificato, creandolo se non già esistente. Restituisce il numero degli
	elementi che risultavano non presenti al momento dell'inserimento (dato soggetto quindi ad
	errore).

	L'errore $P$ può essere indicato opzionalmente solo al momento della creazione, cioè del primo
	inserimento; se viene specificato in un filtro già esistente, e il valore specificato è diverso
	da quello con cui il filtro è stato creato, il comando restituisce un errore senza effettuare
	alcun inserimento.

	\item \textbf{BFEXIST <key> <ele>}: Verifica la presenza di un elemento nel filtro specificato
	(test di appartenenza). Se il filtro non esiste, il comando restituisce falso.

	\item \textbf{BFCOUNT <key>}: Restituisce una stima del numero di elementi inseriti nel filtro.
	Se il filtro non esiste, il comando restituisce $0$.

	\item \textbf{BFDEBUG}: Restituisce informazioni utili al debugging dell'implementazione del
	filtro. Questo comando accetta come primo parametro un sotto-comando; al momento, ne sono stati
	implementati due:

		\begin{itemize}
		\item \textbf{BFDEBUG STATUS <key>}: Restituisce una stringa che contiene informazioni sullo
		stato attuale del filtro, e in particolare il numero di filtri presenti nella catena, e
		l'errore che è stato specificato.

		\item \textbf{BFDEBUG FILTER <key> <idx>}: Restituisce informazioni su uno specifico filtro
		della catena; il filtro che si vuole ispezionare deve essere specificato come indice 
		nel parametro \verb|idx|. La stringa restituita mostra il numero di sezioni (o funzioni
		di hash), la dimensione in bit di una sezione, e il numero di bit impostati a 1.
		\end{itemize}

\end{itemize}

Si noti come, avendo esposto il solo parametro di probabilità, di fatto i comandi si applicano ad
una qualunque struttura dati probabilistica che implementa un insieme, consentendo in futuro di
cambiare implementazione se lo si ritenesse necessario, senza invalidare l'utilizzo da parte dei
client. I comandi di Redis, infatti, mantengono una retrocompatibilità completa fin dalla prima
versione, in modo che sia sempre possibile aggiornare il server senza causare problemi ai client.

\section{Esempio d'uso dei comandi}

Vediamo ora un esempio di sessione utilizzando i comandi appena descritti:

\medskip
\begin{redis}[numbers=left,caption=Esempio di utilizzo dei nuovi comandi per i filtri di Bloom]
127.0.0.1:6379> BFADD prova ELEMENTS a b c d e |\label{line:bfadd}|
(integer) 5
127.0.0.1:6379> BFCOUNT prova |\label{line:bfcount}|
(integer) 5
127.0.0.1:6379> BFADD prova ELEMENTS f g a |\label{line:bfadd2}|
(integer) 2
127.0.0.1:6379> BFCOUNT prova
(integer) 7
127.0.0.1:6379> BFEXIST prova b |\label{line:bfexist}|
(integer) 1
127.0.0.1:6379> BFEXIST prova z
(integer) 0
127.0.0.1:6379> BFDEBUG STATUS prova |\label{line:bfdebugstatus}|
"n:1 e:0.003"
127.0.0.1:6379> BFDEBUG FILTER prova 0 |\label{line:bfdebugfilter}|
"k:11 s:1798 b:77"
\end{redis}

Per cominciare, alla \autoref{line:bfadd} viene creato un filtro chiamato \verb|prova| in cui
vengono inseriti 5 elementi (ciascun elemento è una stringa di un solo carattere, corrispondenti
alle prime cinque lettere dell'alfabeto). Il comando restituisce il valore $5$, corrispondente al
numero di elementi inseriti. Alla \autoref{line:bfcount}, viene effettuato un calcolo della 
cardinalità stimata, che restituisce il valore corretto $5$.

Alla \autoref{line:bfadd2} viene fatto un secondo inserimento; questa volta, oltre a due
elementi nuovi (stringhe \verb|f| e \verb|g|), viene inserito anche un elemento che era stato
già precedentemente inserito: la lettera \verb|a|. Il comando restituisce $2$, riportando
correttamente il numero di nuovi elementi inseriti, e anche il successivo calcolo della
cardinalità è corretto, poiché restituisce $7$

Alla \autoref{line:bfexist} viene effettuato un primo test di appartenenza al filtro tramite
il comando \verb|BFEXIST|, richiedendo
se l'elemento corrispondente alla stringa \verb|b| è presente; l'elemento era stato inserito
precedentemente (\autoref{line:bfadd}), ed infatti il comando restituisce correttamente $1$ per
indicare che l'elemento è presente. Successivamente, viene effettuato un secondo test con
l'elemento \verb|z| che non era stato mai inserito, e il comando questa volta restituisce $0$.

Alla \autoref{line:bfdebugstatus} viene usato il comando \verb|BFDEBUG STATUS| per ispezionare lo
stato del filtro. La stringa restituita è composta da due valori:

\medskip
\begin{tabular}{ |l|l|p{280pt}| }
  \hline
  $n$ & $1$ & La catena del filtro scalabile contiene un solo filtro \\
  $e$ & $0.003$ & La probabilità di falsi positivi è impostata a \SI{0.3}{\percent} \\
  \hline
\end{tabular}
\medskip

Alla \autoref{line:bfdebugfilter}, viene eseguito il comando \verb|BFDEBUG FILTER| per
ispezionare lo stato del primo filtro della catena. La stringa restituita contiene tre valori:

\medskip
\begin{tabular}{ |l|l|p{280pt}| }
  \hline
  $k$ & $11$ & Il filtro utilizza $11$ funzioni di hash, cioè è diviso in $11$ sezioni \\
  $s$ & $1798$ & Ciascuna sezione è formata da \SI{1798}{\bit}, per un totale di $\num{1798 x 11} =
  \SI{19778}{\bit}$ \\
  $b$ & $77$ & Al momento ci sono \SI{77}{\bit} impostati a $1$. Questo corrisponde infatti a $7$
  elementi inseriti con $11$ funzioni di hash, e senza che si sia verificata alcuna collisione. \\
  \hline
\end{tabular}
\medskip

\section{Analisi delle modifiche effettuate}
\label{sec:patchexplain}

Le modifiche effettuate a Redis sono disponibili su GitHub, all'interno del
\href{https://github.com/rasky/redis/tree/bloomfilter}{repositorio fork di Redis}, in un branch
chiamato \verb|bloomfilter|, e la parte principale del codice è riportata anche
nella \autoref{sec:modredis}).

\lstset{
	language=C,
	rangeprefix=/*\ ---------\ ,
	rangesuffix=\ ---------\ */,
	includerangemarker=false,
	escapeinside={/*@}{*/},
	numbers=left,
}

\subsection{Strutture dati}

Nel file \verb|bloom.h| vengono definite le due strutture dati principali:

\lstinputlisting[caption=Strutture dati,linerange=Structures-End\ structures]{code/bloom.h}

La struttura \verb|bloom| rappresenta l'intero filtro scalabile. I suoi campi sono
così definiti:

\medskip
\begin{tabular}{ |l|l|p{280pt}| }
  \hline
  \verb|e| & \verb|double| & La probabilità richiesta di un falso positivo. Questo è il valore richiesto
  dall'utente, che il filtro cercherà di rispettare come margine superiore. \\

  \verb|numfilters| & \verb|int| & Numero di filtri presenti nella catena. \\

  \verb|first| & \verb|filter*| & Puntatore alla testa della lista concatenata di filtri. \\
  \hline
\end{tabular}
\medskip

Come si evince, i filtri sono conservati in una lista a catena singola, e il campo \verb|first|
contiene il puntatore alla testa. La scelta di una lista per l'implementazione della catena di
filtri semplifica l'implementazione ed è compatibile con gli algoritmi che la utilizzano poiché
molto spesso è richiesto di attraversare l'intera catena, mentre per esempio l'accesso casuale non è
mai richiesto.

Ciascun filtro è rappresentato dalla struttura \verb|filter|, definita dai seguenti campi:

\medskip
\begin{tabular}{ |l|l|p{280pt}| }
  \hline

  \verb|next| & \verb|filter*| & Puntatore all'elemento successivo nella lista \\

  \verb|k| & \verb|uint32| & Numero di funzioni di hash (e dunque di partizioni) \\

  \verb|b| & \verb|uint64| & Numero di bit attualmente impostati a $1$, considerando tutte le partizioni \\

  \verb|s| & \verb|uint64| & Dimensione di ciascuna partizione in bit. L'utilizzo di memoria
  effettivo è $\ceil{s/7}$ \\

  \verb|bmax| & \verb|uint64| & Numero di bit che possono essere impostati al massimo a $1$ prima
  che venga raggiunta la densità scelta come obiettivo (tipicamente, \SI{50}{\percent}). \\

  \verb|parts| & \verb|uint8*[]| & Array di puntatori alle partizioni. \\

  \hline
\end{tabular}
\medskip

Si noti che l'array \verb|parts| è definito a dimensione variabile (anche detto \emph{array
flessibile}), utilizzando la funzionalità introdotta nello standard C99 che consente all'ultimo
membro di una struttura, se array, di avere una dimensione non dichiarata staticamente, ma gestita
dinamicamente al momento di allocare la struttura stessa sullo heap. Poiché la dimensione dell'array
è pari a \verb|k| e non cambia mai dopo l'inizializzazione della struttura, questa scelta consente
di allocare la struttura \verb|filter| e l'array di puntatori dinamico con una sola allocazione.

\subsection{Parametri di configurazione statici}

All'inizio del file \verb|bloom.c|, troviamo alcune costanti di configurazione per le quali si è
preferito non esporre una configurazione a livello utente.

\lstinputlisting[caption=Parametri statici,linerange=Static\ parameters-End\ of\ static\ parameters]{code/bloom.c}

Questa la descrizione dettagliata di ciascun parametro:

\medskip
\begin{itemize}
  \item \verb|CONFIG_BLOOM_BASESIZE| (\SI{2048}{\byte}). Dimensione iniziale in byte del primo
  filtro della catena. La dimensione è stata calcolata tenendo conto di un bilanciamento tra uso
  della memoria, crescita del numero di filtri nella catena, e velocità (vedi la
  \autoref{sec:patch:basesize}).

  \item \verb|CONFIG_BLOOM_DEFAULTERROR| ($\SI{0.3}{\percent}$). Probabilità di default di falso
  positivo a cui il filtro scalabile tende asintoticamente. La scelta del valore cerca di
  massimizzare i casi d'uso in cui il filtro può essere usato con successo (vedi la
  \autoref{sec:patch:defaulterror}).

  \item \verb|CONFIG_BLOOM_DESIREDFILLRATIO| ($\SI{50}{\percent}$). Valore di densità massimo dei filtri
  nella catena; quando il filtro in fondo alla catena (l'unico su cui si inseriscono nuovi elementi)
  raggiunge questa densità, un nuovo filtro viene creato. Il valore ottimizza l'uso dei filtri di
  Bloom, come visto nella \autoref{sec:bloomparms}.

  \item \verb|CONFIG_BLOOM_ITEMGROWTHRATIO| ($\num{2.0}$). Fattore di crescita dei filtri
  all'interno della catena. Come visto nella \autoref{sec:bloomscalable} dove è chiamato
  $s$, questo fattore delinea una progressione geometrica che consente di adattare la crescita del
  filtro all'aumentare del numero di elementi inseriti senza creare una catena troppo lunga
  né sprecando troppa memoria.

  \item \verb|CONFIG_BLOOM_TIGHTENINGRATIO| ($\num{0.85}$). Fattore di intensificazione, applicato
  per creare una progressione geometrica tra le probabilità di falsi positivi dei filtri all'interno
  della catena, come descritto nella \autoref{sec:bloomscalable}. Il valore ottimale è indicato
  essere compreso in $\interval{0.8}{0.9}$ in \cite{bloomscalable}.

\end{itemize}
\medskip

\subsection{Creazione e distruzione di un filtro}

La creazione e distruzione di un intero filtro scalabile è implementata dalle seguenti funzioni:

\lstinputlisting[caption=Creazione e distruzione di un filtro scalabile,linerange=Bloom\ constructors-End\ of\ bloom\ constructors]{code/bloom.c}

La funzione \verb|bloomNew| alloca una struttura di tipo \verb|bloom| sullo heap, utilizzando la
funzione \verb|zmalloc|, una funzione interna di Redis che implementa un sottile strato di codice
sopra l'allocatore standard per tracciare l'occupazione totale di memoria e fornire informazioni
utili a livello statistico e per debugging. 

Si noti che al momento di creare il filtro scalabile, alla \autoref{line:noallocfilter} la catena
viene lasciata completamente vuota: ritardando la creazione del primo filtro della catena
all'inserimento del primo elemento, è possibile infatti accorgersi velocemente se, in un dato
momento, è stato mai inserito un elemento, caratteristica utile alla corretta implementazione del
comando \verb|BFADD| che deve discriminare il primo inserimento per permettere la configurazione del
parametro di errore.

La funzione \verb|bloomRelease| rilascia la memoria allocata dall'intero filtro scalabile, percorrendo
la catena e chiamando la funzione di distruzione di ciascun filtro, terminando poi rilasciando la
memoria occupata dalla struttura del filtro stessa.

\lstinputlisting[caption=Creazione e distruzione di un filtro,linerange=Filter\ constructors-End\ of\ filter\ constructors]{code/bloom.c}

La funzione \verb|bloomFilterNew| riceve in input un filtro scalabile e ritorna un filtro appena
creato con i parametri adatti ad essere l'ultimo filtro della catena.

Il calcolo dei parametri del filtro ricalca quanto visto nella \autoref{sec:bloomparms} e nella
\autoref{sec:bloomscalable}. Alla \autoref{line:allocmem}, si può notare come la memoria
allocata tenga conto dell'array flessibile delle sezioni, lasciando lo spazio per il giusto numero
di puntatori (uno per sezione). Nel ciclo successivo, alla \autoref{line:filterloopalloc}, vengono
allocate invece le singole sezioni ed inizializzate a $0$.

\subsection{Aggiunta di un elemento al filtro}

Vediamo ora come viene aggiunto un elemento ad un filtro. Una prima funzione \verb|bloomAdd| si
occupa di individuare il filtro giusto all'interno della catena:

\lstinputlisting[caption=Aggiunta elemento ad un filtro scalabile,linerange=bloomAdd-End\ of\ bloomAdd]{code/bloom.c}

Se la catena è vuota, viene immediatamente creato il primo filtro. Altrimenti, si percorre la catena
fino ad arrivare all'ultimo filtro, l'unico dentro cui è possibile aggiungere elementi (poiché tutti
gli altri avranno già raggiunto la densità massima); se anche esso ha superato la densità massima 
configurata, viene aggiunto un nuovo filtro vuoto in fondo alla catena, usando \verb|bloomFilterNew|.

\lstinputlisting[caption=Aggiunta elemento ad un filtro,linerange=bloomFilterAdd-End\ of\ bloomFilterAdd]{code/bloom.c}

La funzione \verb|bloomFilterAdd| effettua invece l'inserimento vero e proprio, secondo
l'\autoref{alg:bloominsert} visto nella \autoref{sec:bloom:add}.

Per le funzioni di hash, viene utilizzato l'\autoref{alg:ehndoublehash} di doppio hash
avanzato, visto nella \autoref{sec:bloom:hash}. L'algoritmo richiede in partenza due funzioni di
hash ($h_a$ e $h_b$): nell'implementazione viene utilizzata la funzione di hash
\verb|MurmurHash64A|, una variante a \SI{64}{\bit} della funzione MurMurHash2
(\autoref{alg:murmurhash2}). Di questa funzione, è presente già un'implementazione dentro il
codice sorgente di Redis, e poiché l'output è a \SI{64}{\bit}, consideriamo la suddivisione in due
valori a \SI{32}{\bit} come l'output di due funzioni di hash separate, ottenendo così i due valori
\verb|a| e \verb|b| utili come input per l'algoritmo di doppio hash avanzato. La funzione
\verb|MurMurHash64A| viene chiamata con un seme causale, calcolato in fase di sviluppo
(\verb|0xc5fb9af2|).

Il calcolo del doppio hash, in formato iterativo, avviene alla \autoref{line:doublehash}, secondo
quanto visto nell'\autoref{alg:ehndoublehash}. Successivamente, alla
\autoref{line:hashreduction}, viene effettuata una riduzione dell'output della funzione di hash
all'interno del dominio scelto. Invece del classico algoritmo di riduzione $idx = hash \bmod s$ (con
$s$ dimensione in bit della sezione), utilizziamo la formula $idx = hash \times s / 2^{32}$ che
effettua comunque la riduzione richiesta, considerando che $index \in \interval[open
right]{0}{2^{32}}$, ma senza richiedere una divisione, che anche nei moderni processori ha una latenza
elevata (circa $26$ cicli, contro i $4$ di una moltiplicazione). Si veda \cite{lemirereduction} per
i dettagli relativi a questa formula di riduzione.

Alla linea~\label{line:togglebit}, viene impostato a $1$ il bit corrispondente all'indice calcolato,
dopo aver verificato il suo stato corrente e aver potuto incrementare così un contatore dei bit che
cambiando di stato (\verb|nbits|).

\subsection{Scelta della dimensione del primo filtro}
\label{sec:patch:basesize}


\subsection{Scelta della probabilità di falso positivo di default}
\label{sec:patch:defaulterror}



