
% Nessun testo che esce dal paragrafo
\tolerance=10000

% Vari pacchetti utilizzati
\usepackage{interval}  % Gestione intervalli aperti/chiusi
\usepackage{hhline,colortbl,multirow}  % Tabelle con celle colorate e hline potenti
\usepackage[binary-units]{siunitx}  % Unità di misura
\usepackage{algpseudocode}  % Algoritmi in pseudo-codice
\usepackage[chapter]{algorithm}
\usepackage{pifont}  % Numeri cerchiati su sfondo nero
\usepackage{pgf-umlsd} % UML sequence diagrams
\usepackage{needspace} % Evita listati/tabelle orfani
\usepackage[inline]{enumitem} % Più opzioni per le definition list, tra cui l'inline
\usepackage{fancyvrb,newverbs} % Per colorare il bkg dei verbatim
\usepackage{ragged2e} % Per disattivare giustificazione nelle note che contengono URL

% Definisci \EUR come unità di misura con il simbolo dell'euro
\usepackage{eurosym}  
\DeclareSIUnit{\EUR}{\text{\euro}}

% Definesci \req come unità di misura per le richieste al secondo
\DeclareSIUnit\req{req}

% Definisci \ceil per fare il ceiling 
\usepackage{mathtools} 
\DeclarePairedDelimiter{\ceil}{\lceil}{\rceil}

% Supporto per UTF-8 nei listati
% Non usiamo il pacchetto listingutf8 e inputencoding=utf-8 perché
% rallenta la compilazione (di minuti...)
\lstset{literate=
  {á}{{\'a}}1 {é}{{\'e}}1 {í}{{\'i}}1 {ó}{{\'o}}1 {ú}{{\'u}}1
  {Á}{{\'A}}1 {É}{{\'E}}1 {Í}{{\'I}}1 {Ó}{{\'O}}1 {Ú}{{\'U}}1
  {à}{{\`a}}1 {è}{{\`e}}1 {ì}{{\`i}}1 {ò}{{\`o}}1 {ù}{{\`u}}1
  {À}{{\`A}}1 {È}{{\'E}}1 {Ì}{{\`I}}1 {Ò}{{\`O}}1 {Ù}{{\`U}}1
  {ä}{{\"a}}1 {ë}{{\"e}}1 {ï}{{\"i}}1 {ö}{{\"o}}1 {ü}{{\"u}}1
  {Ä}{{\"A}}1 {Ë}{{\"E}}1 {Ï}{{\"I}}1 {Ö}{{\"O}}1 {Ü}{{\"U}}1
  {â}{{\^a}}1 {ê}{{\^e}}1 {î}{{\^i}}1 {ô}{{\^o}}1 {û}{{\^u}}1
  {Â}{{\^A}}1 {Ê}{{\^E}}1 {Î}{{\^I}}1 {Ô}{{\^O}}1 {Û}{{\^U}}1
  {œ}{{\oe}}1 {Œ}{{\OE}}1 {æ}{{\ae}}1 {Æ}{{\AE}}1 {ß}{{\ss}}1
  {ű}{{\H{u}}}1 {Ű}{{\H{U}}}1 {ő}{{\H{o}}}1 {Ő}{{\H{O}}}1
  {ç}{{\c c}}1 {Ç}{{\c C}}1 {ø}{{\o}}1 {å}{{\r a}}1 {Å}{{\r A}}1
  {€}{{\euro}}1 {£}{{\pounds}}1 {«}{{\guillemotleft}}1
  {»}{{\guillemotright}}1 {ñ}{{\~n}}1 {Ñ}{{\~N}}1 {¿}{{?`}}1
}

% Colore di sfondo (definito da xcolor, copiato qui), usato
% per lo sfondo dei listati.
\definecolor{LemonChiffon1}{RGB}{255,250,205}

% Configurazione estetica di default dei listati
\lstset{
  stepnumber=1,
  frame=tb,
  framexleftmargin=0.5em,
  framexrightmargin=0.5em,
  aboveskip=3mm,
  belowskip=3mm,
  columns=flexible,
  stringstyle=\color{red},
  basicstyle=\ttfamily\footnotesize,
  tabsize=4,
  backgroundcolor=\color{LemonChiffon1},
}

% Definzione della colorazione per le sessioni Redis
\lstdefinelanguage{Redis}{
  morekeywords={
    OK, QUEUED,
    SET, GET, DEL, STRLEN, INCR, INCRBY, DECRBY, GETRANGE, APPEND,
    RPUSH, LINDEX, LPOP, LLEN, LINSERT, LRANGE, AFTER,
    HSET, HMSET, HLEN, HGET, HGETALL, HKEYS, HVALS, HEXISTS,
    SADD, SISMEMBER, SREM, SCARD, SDIFF, SINTER,
    ZADD, ZCARD, ZSCORE, ZRANGE, ZREVRANGE, ZRANGEBYSCORE, ZINCRBY, ZREM,
    PFADD, PFCOUNT, 
    SCRIPT, LOAD,
    EVALSHA,
    WATCH, MULTI, EXEC,
    EXPIREAT, TTL, PTTL, PEXPIRE,
    BFADD, BFCOUNT, BFEXIST, ELEMENTS, ERROR, BFDEBUG, STATUS, FILTER,
  }, 
  sensitive=true,
  moredelim=[s][\color{Maroon}\itshape]{127}{>},
  morestring=[b]", 
}

\lstnewenvironment{redis}[1][]
  {\lstset{
    language=Redis,
    escapechar=|,
    #1
  }}{}

\lstdefinestyle{redis}{
    language=Redis,
    escapechar=|,
}

\lstdefinestyle{csource}{
    language=C,
    rangeprefix=/*\ ---------\ ,
    rangesuffix=\ ---------\ */,
    includerangemarker=false,
    escapeinside={/*@}{@*/},
    numbers=left,
}

% Macro per caricare un listato da file, inserendo come caption il suo nome
\newcommand{\sourcecode}[2][]{%
    \lstinputlisting[
      caption={\texttt{\detokenize{#2}}},
      basicstyle=\ttfamily\scriptsize, % un po' più piccolo, visto che è l'intero file
      #1
    ]{#2}%
}

% Sistema di macro per commentare il codice sorgente, aggiungendo dei riferimenti
% progressivi numerici a margine per indicare specifiche linee del codice.
% https://zuttobenkyou.wordpress.com/2010/12/05/latex-saner-source-code-listings-no-starch-press-style/
\newcounter{lstNoteCounter}
\newcommand{\lnnumraw}[1]
    {\raisebox{-.1ex}
    {\ifthenelse{#1 =  1}{\ding{182}}
    {\ifthenelse{#1 =  2}{\ding{183}}
    {\ifthenelse{#1 =  3}{\ding{184}}
    {\ifthenelse{#1 =  4}{\ding{185}}
    {\ifthenelse{#1 =  5}{\ding{186}}
    {\ifthenelse{#1 =  6}{\ding{187}}
    {\ifthenelse{#1 =  7}{\ding{188}}
    {\ifthenelse{#1 =  8}{\ding{189}}
    {\ifthenelse{#1 =  9}{\ding{190}}
    {\ifthenelse{#1 = 10}{\ding{191}}
    {NUM TOO HIGH}}}}}}}}}}}}
\newcommand*{\lnote}[1][1em]{\stepcounter{lstNoteCounter}\vbox{\llap{{\large\lnnumraw{\thelstNoteCounter}}\hskip #1}}}
\newcommand*{\lnnum}[1]{\unskip\,\lnnumraw{#1}}

\lstnewenvironment{commentedsource}[1][]{
    \medskip
    \needspace{5\baselineskip}% se non ci sono almeno 5 linee a disposizione, vai a nuova pagina
    \setcounter{lstNoteCounter}{0}
    \lstset{#1,numbers=none}
}{}

% Definisce comandi \cverb e \ctexttt per codice inline formattato con 
% un leggero fondale grigio che lo evidenzia rispetto al testo normale.
\definecolor{cverbbg}{gray}{0.95}
%\colorlet{cverbbg}{LemonChiffon1}
\newcommand{\ctexttt}[1]{\colorbox{cverbbg}{\texttt{#1}}}
\newverbcommand{\cverb}
  {\setbox\verbbox\hbox\bgroup}
  {\egroup\colorbox{cverbbg}{\box\verbbox}}
\newcommand{\mystrut}{\vrule height6.0pt depth0pt width0pt}
\newverbcommand{\cverbfull} % full height box
  {\setbox\verbbox\hbox\bgroup}
  {\egroup\colorbox{cverbbg}{\box\verbbox\mystrut}}


% Regole custom di sillabazione
\hyphenation{macOS}  % non sillabare mai!

% Mostra le URL come note a pié di pagina, per essere leggibili anche in caso
% di stampa. Derivato da:
% https://tex.stackexchange.com/questions/115479/insert-footnotes-with-href
% con l'aggiunta di \RaggedRight per disattivare la giustificazione e dunque
% evitare i warning di hbox undefull.
\newcommand{\fnhref}[2]{%
  \href{#1}{#2}\footnote{\RaggedRight\url{#1}}%
}

% Spegni warning su inclusione immagini PDF
\ifpdf
  \pdfsuppresswarningpagegroup=1
\fi

% Spegni warning di pdflatex relativo al riuso dei target hyperref
% tra algoritmi diversi. In realtà, al momento questo tipo di target non viene
% neppure mai utilizzato (utilizziamo \lnote e \lnnum definiti sopra), ma
% in questo modo il warning scompare.
% https://tex.stackexchange.com/questions/177025/hyperref-cleveref-and-algpseudocode-same-identifier-warning
\makeatletter
\newcounter{algorithmicH}% New algorithmic-like hyperref counter
\let\oldalgorithmic\algorithmic
\renewcommand{\algorithmic}{%
  \stepcounter{algorithmicH}% Step counter
  \oldalgorithmic}% Do what was always done with algorithmic environment
\renewcommand{\theHALG@line}{ALG@line.\thealgorithmicH.\arabic{ALG@line}}
\makeatother

% Diminuisce le spaziature tra i rule delle tabelle e le cell, così da poterle
% colorare senza spazi bianchi
\setlength{\aboverulesep}{0pt}
\setlength{\belowrulesep}{0pt}
\setlength{\extrarowheight}{.75ex}
