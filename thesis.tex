%--------------------------------------------------------------
% thesis.tex
%--------------------------------------------------------------
% Corso di Laurea in Informatica
% http://if.dsi.unifi.it/
% @Facolt\`a di Scienze Matematiche, Fisiche e Naturali
% @Universit\`a degli Studi di Firenze
%--------------------------------------------------------------
% - template for the main file of Informatica@Unifi Thesis
% - based on Classic Thesis Style Copyright (C) 2008
%   Andr\'e Miede http://www.miede.de
%--------------------------------------------------------------
% **************************************************************************************************************
\RequirePackage{silence} % :-\
    \WarningFilter{scrreprt}{Usage of package `titlesec'}
    \WarningFilter{scrreprt}{Activating an ugly workaround}
    \WarningFilter{titlesec}{Non standard sectioning command detected}
\documentclass[ twoside,openright,titlepage,numbers=noenddot,%1headlines,
                headinclude,footinclude,cleardoublepage=empty,
                BCOR=5mm,paper=a4,fontsize=11pt
                ]{scrreprt}

%********************************************************************
% Note: Make all your adjustments in here
%*******************************************************
\input{classicthesis-config}

% Nessun testo che esce dal paragrafo
\tolerance=10000

% Vari pacchetti utilizzati
\usepackage{interval}  % Gestione intervalli aperti/chiusi
\usepackage{colortbl}  % Tabelle con celle colorate
\usepackage[binary-units]{siunitx}  % Unità di misura
\usepackage{algorithm,algpseudocode}  % Algoritmi in pseudo-codice

% Definisci \EUR come unità di misura con il simbolo dell'euro
\usepackage{eurosym}  
\DeclareSIUnit{\EUR}{\text{\euro}}

% Definisci \ceil per fare il ceiling 
\usepackage{mathtools} 
\DeclarePairedDelimiter{\ceil}{\lceil}{\rceil}

% Supporto per UTF-8 nei listati
% Non usiamo il pacchetto listingutf8 e inputencoding=utf-8 perché
% rallenta la compilazione (di minuti...)
\lstset{literate=
  {á}{{\'a}}1 {é}{{\'e}}1 {í}{{\'i}}1 {ó}{{\'o}}1 {ú}{{\'u}}1
  {Á}{{\'A}}1 {É}{{\'E}}1 {Í}{{\'I}}1 {Ó}{{\'O}}1 {Ú}{{\'U}}1
  {à}{{\`a}}1 {è}{{\`e}}1 {ì}{{\`i}}1 {ò}{{\`o}}1 {ù}{{\`u}}1
  {À}{{\`A}}1 {È}{{\'E}}1 {Ì}{{\`I}}1 {Ò}{{\`O}}1 {Ù}{{\`U}}1
  {ä}{{\"a}}1 {ë}{{\"e}}1 {ï}{{\"i}}1 {ö}{{\"o}}1 {ü}{{\"u}}1
  {Ä}{{\"A}}1 {Ë}{{\"E}}1 {Ï}{{\"I}}1 {Ö}{{\"O}}1 {Ü}{{\"U}}1
  {â}{{\^a}}1 {ê}{{\^e}}1 {î}{{\^i}}1 {ô}{{\^o}}1 {û}{{\^u}}1
  {Â}{{\^A}}1 {Ê}{{\^E}}1 {Î}{{\^I}}1 {Ô}{{\^O}}1 {Û}{{\^U}}1
  {œ}{{\oe}}1 {Œ}{{\OE}}1 {æ}{{\ae}}1 {Æ}{{\AE}}1 {ß}{{\ss}}1
  {ű}{{\H{u}}}1 {Ű}{{\H{U}}}1 {ő}{{\H{o}}}1 {Ő}{{\H{O}}}1
  {ç}{{\c c}}1 {Ç}{{\c C}}1 {ø}{{\o}}1 {å}{{\r a}}1 {Å}{{\r A}}1
  {€}{{\euro}}1 {£}{{\pounds}}1 {«}{{\guillemotleft}}1
  {»}{{\guillemotright}}1 {ñ}{{\~n}}1 {Ñ}{{\~N}}1 {¿}{{?`}}1
}

\definecolor{LemonChiffon1}{RGB}{255,250,205}

% Configurazione estetica di default dei listati
\lstset{
  escapechar=|,
  stepnumber=1,
  frame=tb,
  framexleftmargin=0.5em,
  framexrightmargin=0.5em,
  aboveskip=3mm,
  belowskip=3mm,
  columns=flexible,
  stringstyle=\color{red},
  basicstyle=\ttfamily\footnotesize,
  tabsize=4,
  backgroundcolor=\color{LemonChiffon1},
}

% Definzione della colorazione per le sessioni Redis
\lstdefinelanguage{Redis}{
  morekeywords={
    OK, QUEUED,
    SET, GET, DEL, STRLEN, INCR, INCRBY, DECRBY, GETRANGE, APPEND,
    RPUSH, LINDEX, LPOP, LLEN, LINSERT, LRANGE,
    HSET, HMSET, HLEN, HGET, HGETALL, HKEYS, HVALS, HEXISTS,
    SADD, SISMEMBER, SREM, SCARD, SDIFF, SINTER,
    ZADD, ZCARD, ZSCORE, ZRANGE, ZREVRANGE, ZRANGEBYSCORE, ZINCRBY, ZREM,
    PFADD, PFCOUNT, 
    SCRIPT, LOAD,
    EVALSHA,
    WATCH, MULTI, EXEC,
    EXPIREAT, TTL, PTTL, PEXPIRE,
    BFADD, BFCOUNT, BFEXIST, ELEMENTS, ERROR, BFDEBUG, STATUS, FILTER,
  }, 
  sensitive=true,
  moredelim=[s][\color{Maroon}\itshape]{127}{>},
  morestring=[b]", 
}
\lstnewenvironment{redis}[1][]
  {\lstset{language=Redis,#1}}{}

% Macro per caricare un listato inserendo come caption il suo nome
\newcommand{\sourcecode}[2][]{%
    \lstinputlisting[
      caption={\texttt{\detokenize{#2}}},
      basicstyle=\ttfamily\scriptsize, % un po' più piccolo, visto che è l'intero file
      #1
    ]{#2}%
}

%********************************************************************
% Translations
%*******************************************************
\makeatletter
\@ifpackageloaded{babel}{%
  \newcaptionname{italian}{\lstlistingname}{Listato}
  \newcaptionname{italian}{\lstlistlistingname}{Elenco dei listati}
  \newcaptionname{italian}{\ALG@name}{Algoritmo}
  \newcaptionname{italian}{\listalgorithmname}{Elenco degli algoritmi}
}{\relax}
\makeatother

\renewcommand{\algorithmicrequire}{\textbf{Precondizione:}}

% Consente \autoref anche sulle linee
% https://tex.stackexchange.com/questions/351225/create-a-smart-cross-reference-to-a-line-in-an-algorithmic-environment-using/351228
\makeatletter
  \patchcmd{\ALG@step}{\addtocounter{ALG@line}{1}}{\refstepcounter{ALG@line}}{}{}
  \newcommand{\ALG@lineautorefname}{Linea}
\makeatother
\newcommand{\algorithmautorefname}{Algoritmo}%

% \autoref per le linee nei listati
\newcommand{\lstnumberautorefname}{Linea}

%********************************************************************
% Bibliographies
%*******************************************************
\addbibresource{Bibliography.bib}


%********************************************************************
% Hyphenation
%*******************************************************
%\hyphenation{put special hyphenation here}

% ********************************************************************
% GO!GO!GO! MOVE IT!
%*******************************************************
\begin{document}
\frenchspacing
\raggedbottom
\selectlanguage{italian} % american ngerman
%\renewcommand*{\bibname}{new name}
%\setbibpreamble{}
\pagenumbering{roman}
\pagestyle{plain}
%********************************************************************
% Frontmatter
%*******************************************************
%\include{FrontBackmatter/DirtyTitlepage}
%*******************************************************
% Titlepage
%*******************************************************
\begin{titlepage}
    % if you want the titlepage to be centered, uncomment and fine-tune the line below (KOMA classes environment)
    \begin{addmargin}[-1.1cm]{-3.1cm}
    \begin{center}
        \large
        \hfill
        \vfill

        \begingroup
            \includegraphics[scale=0.15]{img/LOGO}\\
%           \spacedallcaps{\myUni} \\
            \myFaculty \\
            \myDegree \\
            \vspace{0.5cm}
            \vspace{0.5cm}
            Tesi di Laurea        
        \endgroup
        
        \vfill
        \begingroup
            \color{Maroon}\spacedallcaps{\myTitle} \\ $\ $\\
            \spacedallcaps{\myEnglishTitle} \\
            \bigskip
        \endgroup

        \spacedlowsmallcaps{\myName}
        \vfill
        \vfill
        Relatore: \emph{\myProf}\\
        \vfill
        \vfill
        \myTime
        \vfill
    \end{center}
  \end{addmargin}
\end{titlepage}

\thispagestyle{empty}

\hfill

\vfill

\noindent\myName: \textit{\myTitle}, \myDegree,
\myTime

\bigskip

\noindent\spacedlowsmallcaps{Relatore}: \\
\myProf

\medskip

\noindent\spacedlowsmallcaps{Università}: \\
\myUni \\
\myFaculty
%
%\medskip
%
%\noindent\spacedlowsmallcaps{Time Frame}: \\
%\myTime

\cleardoublepage%*******************************************************
% Dedication
%*******************************************************
\thispagestyle{empty}
%\phantomsection
\refstepcounter{dummy}
\pdfbookmark[1]{Dedica}{Dedica}

\vspace*{3cm}

\begin{center}
    Al nonno Enzo, che tanto avrebbe voluto leggerla \\ \smallskip
    1924\,--\,2005
\end{center}

%\cleardoublepage\include{FrontBackmatter/Foreword}
%\cleardoublepage\include{FrontBackmatter/Abstract}
%\cleardoublepage\include{FrontBackmatter/Publications}
\cleardoublepage%*******************************************************
% Acknowledgments
%*******************************************************
\pdfbookmark[1]{Ringraziamenti}{ringraziamenti}

\begin{flushright}
{
	\emph{
		Ex quo exsistit et illud multa esse probabilia, quae, \\
		quamquam non perciperentur, tamen, \\
		quia visum quendam haberent insignem et inlustrem, \\
		his sapientis vita regeretur. \\
	}

	\medskip

	Ne deriva che vi sono delle conoscenze probabili le quali, \\
	benché non possano essere compiutamente accertate, \\
	appaiono così nobili ed elevate \\
	da poter fungere da guida per il saggio. \\

	\medskip

	\textbf{-- Cicerone, De Natura Deorum, I, 12} 
}
\end{flushright}

\bigskip

\begingroup
\let\clearpage\relax
\let\cleardoublepage\relax
\let\cleardoublepage\relax
\chapter*{Ringraziamenti}

Many thanks to everybody who already sent me a postcard!

Regarding the typography and other help, many thanks go to Marco
Kuhlmann, Philipp Lehman, Lothar Schlesier, Jim Young, Lorenzo
Pantieri and Enrico Gregorio\footnote{Members of GuIT (Gruppo
Italiano Utilizzatori di \TeX\ e \LaTeX )}, J\"org Sommer,
Joachim K\"ostler, Daniel Gottschlag, Denis Aydin, Paride
Legovini, Steffen Prochnow, Nicolas Repp, Hinrich Harms,
Roland Winkler, Jörg Weber, Henri Menke, Claus Lahiri,
Clemens Niederberger, Stefano Bragaglia, Jörn Hees,
Scott Lowe, Dave Howcroft, Jos\'e M. Alcaide, 
and the whole \LaTeX-community for support, ideas and
some great software.

\bigskip

\noindent\emph{Regarding \mLyX}: The \mLyX\ port was intially done by
\emph{Nicholas Mariette} in March 2009 and continued by
\emph{Ivo Pletikosi\'c} in 2011. Thank you very much for your
work and for the contributions to the original style.


\endgroup

\cleardoublepage\include{FrontBackmatter/Contents}
%********************************************************************
% Mainmatter
%*******************************************************
\cleardoublepage
\pagestyle{scrheadings}
\pagenumbering{arabic}
%\setcounter{page}{90}
% use \cleardoublepage here to avoid problems with pdfbookmark
\cleardoublepage
\chapter{Introduzione}

Questa tesi presenta una modifica a Redis, un popolare database NoSQL, con la quale vengono
introdotti i filtri di Bloom quale struttura dati aggiuntiva a quelle già presenti. In particolare,
l'aggiunta è utile per arricchire il sempre crescente numero di strutture dati che Redis offre agli
utilizzatori come strumento di catalogazione dell'informazione.

Sono sempre stato interessato alle strutture dati probabilistiche fin da quanto lo ho scoperte; il
concetto di poter arbitrariamente disfarsi di una parte di informazione mantenendone un sottoinsieme
che, per quanto non sufficiente a ricostruire l'intero set di dati, è comunque sufficiente ad alcuni
scopi, è sempre stato per me affascinante, in quanto strettamente legato all'ottimizzazione e
l'efficienza degli algoritmi.

L'obiettivo di questo lavoro è stato quindi quello di approfondire un database NoSQL che ho avuto
modo di utilizzare, Redis, studiandone più da vicino il funzionamento, e contemporaneamente
potenziandone le funzionalità. Redis implementa già ad oggi una struttura dati probabilistica
(HyperLogLog), e avendo trovato riferimenti dell'interesse dell'autore ad espandere il progetto in
questo senso, ho ritenuto che fosse un percorso di ricerca adatto alle mie competenze e alla mia
passione.

La tesi è articolata in cinque capitoli: oltre a questa introduzione, nel secondo capitolo viene
descritto il funzionamento e l'architettura di Redis, descrivendo il contesto scientifico e
industriale che ne ha portato alla nascita, gli scenari applicativi principali in cui viene
utilizzato, le strutture dati che offre, le funzionalità più avanzate e le caratteristiche di
persistenza. Nel terzo capitolo invece vengono descritti i filtri di Bloom, studiandone gli scenari
applicativi, le principali caratteristiche, gli algoritmi e analizzando nel dettaglio la loro
parametrizzazione. Nel quarto capitolo, viene presentato il lavoro svolo di modifica al codice
sorgente di Redis, sottolineando le principali sfide incontrate e i risultati raggiunti in termini
di funzionalità e performance. Nel quinto ed ultimo capitolo, si commentano i risultati ottenuti e
si descrive una possibile evoluzione del lavoro.

Il lavoro di questa tesi è disponibile online ed è stato inviato agli autori di Redis per
l'integrazione nella base di codice ufficiale.

\chapter{Redis: un veloce database in memoria}

\section{I database NoSQL}

Per database NoSQL, si intende ogni software atto alla memorizzazione
strutturata dell'informazione che, in rottura con l'usanza predominante già dagli anni
'70, non utilizza lo \emph{Structured Markup Langage} (SQL) per la manipolazione
dei dati ivi contenuti. Nella maggioranza dei casi, l'assenza del linguaggio SQL in
realtà è solamente un sintomo di una importante differenza architetturale: l'allontanamento
dal paradigma di strutturazione in tabelle, record e relazione, e, se vogliamo, anche dal modello E-R.
Da questo punto di vista, il modo più corretto per identificare l'insieme di questi
software sarebbe quindi l'uso della locuzione "database non relazionali".

Il termine si è diffuso nel linguaggio popolare informatico dal 2010; si suole far
coincidere questa improvvisa attenzione con un omonimo workshop organizzato dalla società
californiana Rackspace (un fornitore di servizi cloud) in cui vennero analizzate
queste tecnologie a seguito del crescente interesse nell'ambiente della cosiddetta
\emph{Silicon Valley}.

I database NoSQL coprono quindi un ampio spettro di software completamente eterogenei
tra loro, adatti a scopi e situazioni diverse, con l'unico tratto a comune di non
utilizzare il modello relazionale.

\subsection{Tipologie di database}

Posto che una tassonomia esaustiva dei database NoSQL sarebbe impossibile in virtù
dell'ampiezza della definizione, ne proponiamo una \cite{corbellini} sufficiente a coprire le
principali tipologie e sottolinearne le differenze:

\begin{itemize}
	\medskip
	\item
	\textbf{Database Chiave-valore}: questi database si comportano come giganteschi array
	associativi distribuiti, nei quali dunque è possibile risalire ad un valore data la
	sua chiave univoca di riferimento. Vengono spessi offerti più spazi di chiavi
	a disposizione, e l'obiettivo di scala è molto elavato (fino a petabyte di dati
	e milioni di operazioni al secondo).

	\item
	\textbf{Database a famiglie di colonne}: questi database memorizzano i dati in un formato
	tabellare, ma senza garantire uno schema; in altre parole, ciascuna riga contiene
	una tupla in cui non tutte le colonne sono valorizzate (e, a seconda dei casi, la
	valorizzazione di una colonna può anche non avere un tipo specifico). In alcuni
	casi, sono presenti delle colonne speciali per identificatori univoci o timestamp,
	su cui costruire per lo meno indici parziali.

	\item
	\textbf{Database orientati al documento}: questi database memorizzano dati organizzati in
	"documenti", indicizzati con una chiave primaria. Ogni documento ha un suo spazio
	chiavi, e viene memorizzato secondo uno schema ben definito, ma più flessibile
	di quello utilizzato nelle tabella dei database relazionali; tipicamente infatti,
	è possibile aggiungere liberamente dati ad un documento, sebbene con determinati
	vincoli.

	\item
	\textbf{Database orientati ai grafi}: questi database sono pensati per memorizzare dati
	in formato tabellare, ma con relazioni multiple tra loro codificate in modo
	strutturato e preciso, in modo da formare dei grafi di relazioni, e con operazioni
	efficienti per effettuare l'analisi di questi grafi. I tipici casi d'uso sono
	quelli dove i dati presentano numerose relazioni di interconnessione, quali per
	esempio i dati di un social network.
\end{itemize}

Poiché Redis, oggetto di questa tesi, rientra pienamente nella prima categoria, sarà
questa che andremo ad analizzare con maggior dettaglio.


\subsection{I database chiave-valore}

Nei database chiave-valore, sia l'inserimento che la ricerca avviene tramite
la chiave primaria. Su questa chiave (tipicamente una stringa o array di byte),
viene applicata internamente una funzione hash che consente poi una strutturazione
in tabella dei dati con ricerca e inserimento efficiente. Per gestire il partizionamento
dei dato su più nodi, si utilizza normalmente una funzione hash consistente.

Per effettuare una ulteriore analisi più approfondita, è necessario suddividere
questi database in due grossi gruppi:

\begin{itemize}
	\medskip
	\item
	\textbf{Database chiave-valore in RAM}. Si tratta di database in cui l'intero dataset
	deve essere necessariamente contenuto nella RAM dei nodi del database
	(con eventuale distribuzione in più nodi). In questi database, dunque, tutte
	le operazioni principali vengono effettuate direttamente in RAM, e la persistenza
	su disco a volte è addirittura opzionale o eventuale (cioè non sempre consistente).
	Anche laddove viene richiesta una persistenza consistente, il database mantiene
	tutti i dati in RAM. Redis rientra in questa categoria.

	\item
	\textbf{Database chiave-valore su disco}. Al contrario dei precedenti, questi
	database seguono una struttura più classica. I dati vengono memorizzati infatti
	su disco fisso, mentre la RAM viene usata come memoria volatile per memorizzare
	parti di dati più frequentemente utilizzati, o indici sui dati stessi per un
	accesso rapido.
\end{itemize}

\section{Nascita dei database chiave-valore in RAM}

I database chiave-valore in RAM sono anche chiamati "cache distribuite", una sineddoche
che rimanda al tipo d'uso più comune, nato intorno alla metà degli anni 2000.

Con l'avvento dell'adozione di massa delle connessioni Internet in banda larga nei primi
anni 2000, i servizi su Internet hanno iniziato a sperimentare dei problemi di stabilità,
quando si trovavano a gestire alti picchi di traffico. Tali servizi, infatti, erano
comunemente implementati come un semplice servizio di tipo CGI, ospitato all'interno
del processo del webserver; in caso di traffico troppo elevato, l'unica soluzione possibile
era quella della cosidetta ``scalabilità verticale'', cioè eseguire il servizio su un
server più potente.

Il tema dei picchi di traffico, così alti e improvvisi da mettere fuori uso i siti,
era così dibattuto da avere anche un nome preciso: era comunemente chiamato ``Slashdot
effect'', dal nome del famoso portale Slashdot che, all'epoca, era molto visitato dagli
informatici di tutto il mondo. Il sito, ancora oggi operativo (sebbene non più così
visitato), pubblica una raccolta curata di notizie e link dedicati al mondo della tecnologia
e delle scienza. Accadeva spesso che alcuni siti, quando venivano pubblicati su Slashdot,
ricevessero una mole di visite troppo elevate che di fatto li mandava fuori uso; da qui
il nome.

In ottica di pura scalabilità verticale, uno dei modi più comuni per ottimizzare il software
applicativo è quello di aggiungere uno strato di cache in RAM, anteposta quindi al database,
per diminuire il numero di query che vengono fatte. Un esempio classico è la gestione delle
sessioni: quando un client comunica via HTTP con il server, inserisce sempre nelle richieste
un cookie che identifica univocamente la sessione dell'utente; tramite il cookie, il server
può quindi riconoscere la sessione e l'utente stesso. Tantissime richieste HTTP contengono un
cookie di sessione, e di conseguenza il server applicativo deve sempre risalire alla sessione
prima ancora di entrare nel merito della richiesta. Nei primi anni 2000, era normale memorizzare
le sessioni nel database, ed era dunque necessaria una query per ogni richiesta per recuperarla.
Per ottimizzare la velocità del software, era quindi opportuno cercare di limitare queste query, per esempio inserendo in una cache in RAM le sessioni usate più di recente.

Il dibattito sulle soluzioni da adottare per ottenere la cosiddetta ``scalabilità orizzontale'',
cioè l'utilizzo di più di un server in parallelo come modo per aumentare le performance,
si orientò velocemente verso la replica indipendente dei server database
e dei server applicativi, anteponendo a questi ultimi dei bilanciatori di carico, che
si occupassero di distribuire le connessioni ingresso. Con questa architettura però,
non era più possibile operare semplicemente una cache in RAM dentro il server applicativo, poiché i bilanciatori di carico (a meno di complesse implementazioni) non potevano facilmente garantire
che ogni richiesta di una stessa sessione arrivasse allo stesso server.

Per ovviare a questo inconveniente, nel 2003 nasce Memcached, scritto da Brad Fitzpatrick per implementare la scalabilità orizzontale nel suo popolare gestore di blog LiveJournal.
Memcached è di fatto una tabella hash chiave-valore, raggiungibile via TCP con un
semplice protocollo testuale. Le operazioni sono stanzialmente due: GET e SET. È sufficiente
qundi collegare tutti i server applicativi ad un unico server Memcached (possibilmente
sulla stessa rete fisica, beneficiando quindi di una latenza molto bassa) perché possano
condividere la stessa cache, aggiornando il codice che prima utilizzava una tabella in RAM
dentro il processo con una libreria che acceda alla stessa tabella tramite chiamate TCP.

Memcached è il primo database NoSQL chiave-valore, ante-litteram, ed è oggi in uso presso alcuni
dei più popolari siti al mondo quali YouTube, Reddit, Facebook, Twitter, Tumblr, Wikipedia.


\section{Nascita di Redis}

Redis (acronimo di Remote Dictionary Service) è un database chiave-valore in RAM scritto
da Salvatore Sanfilippo e rilasciato per la prima volta nel 2008 sotto licenza BSD.

Redis nasce come costola di un progetto (ora defunto) chiamato "lloogg", un servizio di
analisi in tempo reale di log di siti. Originariamente, Sanfilippo aveva progettato
questo servizio  utilizzando MySQL come database primario per la memorizzazione dei
dati, ma presto questa scelta si era dimostrata errata, perché MySQL non raggiungeva le
performance desiderata in un contesto particolarmente intensivo dal punto di vista delle
scrittura come quello degli aggregatori di log \cite{nascita}.

Sanfilippo aveva necessità di memorizzare velocemente i dati in arrivo, e di eseguire
delle semplice query strutturate tipo "estrai gli ultimi N dati inseriti". Questo genere
di struttura non si applica bene al modello relazionale, in particolare perché l'ordine
di inserimento dei dati non viene preservato, e richiede quindi un'operazione di ordinamento
(o aggiornamento di un indice di ordinamento) ogni volta che un dato viene inserito.
Abbandonando il modello relazionale, invece, operazioni di questo genere si eseguono
in modo naturale ed efficiente con strutture dati quali le liste concatenate, ed è
proprio questa impedenza fondamentale tra modello relazionale e strutture dati primarie
alla base dell'architettura di Redis.

\section{Principali casi d'uso}


\section{Installazione}

Redis è stato scritto per funzionare su sistemi UNIX e supporta ufficialmente la
compilazione sotto Linux, macOS, OpenBSD, NetBSD e FreeBSD. Il codice è stato scritto
per essere compatibile anche con architetture a 32-bit, sebbene la versione a 64-bit
sia di gran lunga la più utilizzata in virtù della maggiore capacità di indirizzamento
di memoria.

La compilazione da codice sorgente è molto semplice: Redis è scritto nel linguaggio
ANSI C90, e le pochissime dipendenze a livello di librerie sono distribuite con il
codice sorgente stesso, quindi è sufficiente disporre sul sistema del compilatore GCC.
Una volta scaricato l'archivio di codice sorgente dalla \cite{pagina di scaricamento ufficiale}
o tramite \verb|git| dal \cite{repositorio ufficiale su GitHub}, è sufficiente lanciare il
comando \verb|make|. Questa l'intera sequenza:

\medskip
\begin{lstlisting}
$ git clone https://github.com/antirez/redis
$ cd redis
$ make
\end{lstlisting}

In alternativa, è possibile installare Redis da un pacchetto di distribuzione binario
presente nella maggior parte dei sistemi operativi. Per esempio, in un sistema Debian
o Ubuntu, è possibile eseguire il comando \verb|apt-get install redis-server|, mentre
su un sistema macOS si può utilizzare il gestore di pacchetti ``Homebrew'' tramite il
comando \verb|brew install redis|.

\section{Architettura}

Redis implementa un array associativo (tramite tabella hash) in cui le chiavi sono
stringhe, e i valori associati sono oggetti di vari tipi (stringhe o strutture dati).
Per comunicare con Redis, un client si deve connettere via TCP (la porta di default è la 6379) e
comunicare tramite un protocollo testuale di tipo master/slave. È possibile utilizzare
anche semplicemente \verb|telnet| per sperimentare, sebbene sia più comodo utilizzare il
client dedicato \verb|redis-cli| che offre il completamento intelligente dei comandi, la
storia dei comandi inseriti, e la formattazione leggibile dei risultati.

Come esempio, la seguente sessione mostra come scrivere e leggere un valore di tipo
stringa:

\medskip
\begin{lstlisting}
127.0.0.1:6379> SET abc provavalore
OK
127.0.0.1:6379> GET abc
"provavalore"
127.0.0.1:6379> GET abcd
(nil)
\end{lstlisting}

Il valore di ritorno visualizzato dalla console è fortemente tipizzato nel protocollo;
nei precedenti casi, il valore di ritorno è sempre una stringa (compreso l'ultimo
caso in cui la stringa è vuota, e viene dunque visualizzata come ``<nil>``).

Il comando \verb|DEL| può essere utilizzato per cancellare un oggetto di tipo
arbitrario:

\medskip
\begin{lstlisting}
127.0.0.1:6379> DEL abc
(integer) 1
127.0.0.1:6379> DEL notexisting
(integer) 0
\end{lstlisting}

In questo caso, il valore di ritorno è un intero, dal quale si può desumere se il comando
ha effettivamente cancellato un oggetto o meno. Gli errori sono anch'essi tipizzati in
modo separato. Per esempio, un comando inesistente restituisce un codice di errore:

\medskip
\begin{lstlisting}
127.0.0.1:6379> MYSQL
(error) ERR unknown command 'MYSQL'
\end{lstlisting}

Ogni chiave in Redis è una stringa (sequenza di byte) di lunghezza arbitraria; non è
necessario che sia stampabile, sebbene questo sia consigliabile per facilitare il
debugging. La funzione hash con la quale la stringa viene convertita in indice è
una semplice funziona polinomiale utilizzante un numero primo come base, 
comunemente chiamata djb2 \cite{djbhash}, dal nome del suo ideatore (Daniel J. Bernestein).

\section{Tipologie di oggetti memorizzabili}

Ogni oggetto memorizzato in Redis deve avere un tipo specifico, che viene serializzato
nella tabella e controllato ad ogni accesso. A livello di protocollo di comunicazione,
ogni comando opera su oggetti di uno specifico tipo; per esempio, i comandi \verb|SET|
e \verb|GET| visti nell'esempio precedente operano solamente su stringhe.

Oltre ai tipi base (stringhe, interi, numeri in virgola mobile), gli oggetti memorizzati
possono essere delle vere e proprio strutture dati quali liste, tabelle hash, insiemi
ordinati e così via. È proprio questa ricchezza in termini di strutture dati che caratterizza
Redis rispetto ad altri database della stessa tipologia.

Redis non prevede comandi espliciti di creazione di oggetti; è normalmente sufficiente
eseguire un comando di modifica di un oggetto (come per esempio il comando per aggiungere
un elemento ad un insieme) su una chiave non utilizzata per creare automaticamente una
struttura dati del tipo specificato, associandola alla chiave. Per esempio, il comando
\verb|SADD| viene utilizzato per aggiungere uno o più elementi ad un insieme; di
conseguenza, il comando \verb|SADD test1 elem1 elem2| creerà automaticamente un insieme 
associato alla chiave \verb|test1|, contenente i due elementi \verb|elem1| e \verb|elem2|.
Se \verb|test1| esistesse già nella base di dati ma fosse di tipo diverso (per esempio,
una lista), il comando \verb|SADD| ritornerebbe un errore (\verb|WRONGTYPE|).

Viceversa, come già visto nel precedente paragrafo, è previsto un comando di cancellazione
generale (\verb|DEL|) che rimuove la chiave e l'oggetto ad essa associato dal database,
qualunque sia il tipo dell'oggetto stesso.


\subsection{Stringhe}

Si tratta di sequenze di byte di lunghezza arbitraria. I comandi
più comuni per manipolarle sono, come visto, \verb|SET| e \verb|GET|, ma anche \verb|APPEND|
per concatenare, \verb|STRLEN| per leggere la lunghezza, o \verb|GETRANGE| per estrarre una
sottostringa.

\medskip
\begin{lstlisting}
127.0.0.1:6379> SET foo abcd
OK
127.0.0.1:6379> GET foo
"abcd"
127.0.0.1:6379> APPEND foo ef
(integer) 6
127.0.0.1:6379> STRLEN foo
(integer) 6
127.0.0.1:6379> GETRANGE foo 2 3
"cd"
\end{lstlisting}

Si noti come \verb|APPEND| ha restituito la nuova lunghezza della stringa, sfruttando
la disponibilità nel protocollo del valore di ritorno come modo per veicolare una
informazione possibilmente utile.

Le stringhe sono in realtà l'unico tipo primario supportato da Redis, ma vengono spesso usate come tipo
debole, cioè è possibile manipolarle come se fossero oggetti di altri tipi. Infatti,
i comandi \verb|INCR|, \verb|DECR|,  \verb|INCRBY| e \verb|DECRBY| ci permettono
di interagire con oggetti di tipo stringa che rappresentano degli interi in base 10:

\medskip
\begin{lstlisting}
127.0.0.1:6379> SET foo 123
OK
127.0.0.1:6379> STRLEN foo
(integer) 3
127.0.0.1:6379> INCR foo
(integer) 124
127.0.0.1:6379> STRLEN foo
(integer) 3
127.0.0.1:6379> DECRBY foo 24
(integer) 100
127.0.0.1:6379> GET foo
"100"
\end{lstlisting}

In questo esempio, alla chiave \verb|foo| è stato associato una stringa che contiene
la rappresentazione di un numero in base 10; il comando \verb|STRLEN| ci rassicura
che si tratta di una stringa di lunghezza 3. Nonostante questo, i comandi \verb|INCR|
e \verb|DECRBY| sono in grado di manipolare la stringa come se fosse un intero (e il
loro valore di ritorno infatti è di tipo intero). Al termine di queste operazioni
matematiche, il comando \verb|GET| conferma che si tratta sempre di una stringa.

Allo stesso modo, viene fornito un supporto molto blando per i numeri in virgola
mobile tramite il comando \verb|INCRBYFLOAT|.

Una possibilità più avanzata è quella di considerare la stringa come un array di bit
utilizzando i comandi \verb|BITCOUNT|, \verb|SETBIT| e \verb|GETBIT| per lavorare
su ciascun bit.

\subsection{Liste}

In Redis, le liste sono strutture dati assimilabili, come complessità computazionale,
alle normali liste con puntatore doppio; i valori contenuti nella lista sono stringhe.
L'accesso al primo e all'ultimo elemento ha una complessità di O(1), mentre
l'accesso all'elemento N richiede O(N). I comandi principali per manipolare una
lista sono i seguenti:

\begin{itemize}
	\medskip
	\item \textbf{LPUSH}. Aggiunge un elemento in cima alla lista.
	\item \textbf{RPUSH}. Aggiunge un elemento in fondo alla lista.
	\item \textbf{LPOP}. Rimuove un elemento dalla cima della lista.
	\item \textbf{RPOP}. Rimuove un elemento dal fondo della lista.
	\item \textbf{LLEN}. Restituisce la lunghezza della lista.
	\item \textbf{LINDEX}. Restituisce l'elemento K-esimo della lista.
	\item \textbf{LRANGE}. Restituisce tutti gli elementi all'interno di un range di indici.
	\item \textbf{LSET}. Modifica l'elemento K-esimo della lista.
	\item \textbf{LINSERT}. Inserisce un elemento prima o dopo un elemento pivot.
\end{itemize}

Qui si può vedere una sessione interattiva di esempio di uso di una lista:

\medskip
\begin{lstlisting}
127.0.0.1:6379> RPUSH mylist mickey minnie goofy donald
(integer) 4
127.0.0.1:6379> LINDEX mylist 0
"mickey"
127.0.0.1:6379> LINDEX mylist 1
"minnie"
127.0.0.1:6379> LPOP mylist
"mickey"
127.0.0.1:6379> LLEN mylist
(integer) 3
127.0.0.1:6379> LINSERT mylist AFTER goofy walt
(integer) 4
127.0.0.1:6379> LRANGE mylist 0 5
1) "minnie"
2) "goofy"
3) "walt"
4) "donald"
\end{lstlisting}

Si noti che non è necessario inizializzare la lista in nessun modo: il primo riferimento alla chiave
\verb|mylist| tramite un comando \verb|RPUSH| (che opera esclusivamente su liste) è sufficiente
per creare dinamicamente l'oggetto di tipo giusto.

\subsection{Tabelle hash}

Una tabella hash consente di memorizzare una serie di coppie di stringhe (denominate ``campo'' e ``valore''),
associandole ad una chiave del database; questo consente quindi di serializzare dei semplici oggetti, sebbene
non sia possibile avere tipologie più complesse di dati.

Un esempio di uso di tabella hash è la memorizzazione delle informazioni sugli utenti di un applicativo; è
possibile infatti utilizzare come chiave una stringa composta da un prefisso (per esempio \verb|user|), seguito
dall'ID univoco (chiave primaria) dell'utente, per esempio un intero progressivo, opzionalmente divis da un
separatore per leggibilità (\verb|user:88|). All'interno della tabella, i campi sono i nomi degli attributi che
si vuole memorizzare (\verb|nome|, \verb|cognome|, ecc.), associati ai relativi valori. 

Si noti che non esiste nessun controllo d'integrità che ci assicuri che le tabelle associate a tutti gli utenti
(tutte le chiavi con prefisso \verb|user|, nel nostro esempio) siano consistenti tra loro e dunque popolate dai
medesimi campi; così come non è possibile verificare la tipizzazione dei campi, poiché i valori sono esclusivamente
di tipo stringa e dunque senza tipo. Se dunque è necessaria uniformità, l'onere è lasciato al client.

I comandi principali sono i seguenti, caratterizzata dal prefisso \verb|H|:

\begin{itemize}
	\medskip
	\item \textbf{HSET}. Aggiunge un campo alla tabella.
	\item \textbf{HGET}. Legge il valore di un campo della tabella.
	\item \textbf{HDEL}. Cancella un campo dalla tabella.
	\item \textbf{HEXISTS}. Verifica se un campo è presente nella tabella.
	\item \textbf{HLEN}. Restituisce il numero di campi presenti nella tabella.
	\item \textbf{HGETALL}. Restituisce tutte i campi e tutti i valori nella tabella.
	\item \textbf{HMGET} e \textbf{HMSET}. Rispettivamente come \verb|HGET| e \verb|HSET| ma operano su campi multipli.
	\item \textbf{HKEYS} e \textbf{HVALS}. Restituiscono rispettivamente tutti i campi e tutti i valori nella tabella.
\end{itemize}

Qui si può vedere una sessione interattiva di esempio di uso di una tabella hash:

\medskip
\begin{lstlisting}
127.0.0.1:6379> HSET user:88 name giovanni
(integer) 1
127.0.0.1:6379> HMSET user:88 surname bajo age 38
OK
127.0.0.1:6379> HLEN user:88
(integer) 3
127.0.0.1:6379> HGET user:88 surname
"bajo"
127.0.0.1:6379> HGETALL user:88
1) "name"
2) "giovanni"
3) "surname"
4) "bajo"
5) "age"
6) "38"
127.0.0.1:6379> HKEYS user:88
1) "name"
2) "surname"
3) "age"
127.0.0.1:6379> HVALS user:88
1) "giovanni"
2) "bajo"
3) "38"
127.0.0.1:6379> HEXISTS user:88 name
(integer) 1
127.0.0.1:6379> HEXISTS user:88 giovanni
(integer) 0
\end{lstlisting}

Come si può vedere, la tabella hash è creata dinamicamente al primo uso, e non è necessario dimensionarla in alcun
modo; Redis implementa in modo efficiente le tabelle hash e le ridimensiona automaticamente al crescere degli
elementi contenuti, mantentendo un fattore massimo di riempimento che garantisce una buona velocità di accesso.

\subsection{Insiemi}

Un insieme in Redis è una sequenza non ordinata di stringhe univoche, associate ad una chiave del database. È
possibile compiere le operazioni base quali l'aggiunta di una stringa, la rimozione di una stringa, il conteggio
della cardinalità, la verifica della presenza di una stringa, l'estrazione di una stringa scelta casualmente
tra quelle presenti. Inoltre, sono previste anche operazioni che coinvolgono più di un insieme, quali il 
calcolo dell'intersezione, dell'unione, della differenza.

Riprendendo il precedente esempio di tabella hash contenente gli utenti, un applicativo potrebbe voler memorizzare
all'interno di un insieme l'elenco degli utenti che soddisfano una determinata proprietà, per esempio tutti
coloro che hanno effettuato almeno un pagamento, o coloro che hanno aggiornato all'ultima versione dell'applicativo. 
In questo caso, si potrebbe quindi creare due insiemi, associati alle chiavi \verb|paying_users| e \verb|updated_users|,
e inserirvi gli ID degli utenti (\verb|88|, \verb|4|, \verb|37|, ecc.). Sarebbe poi possibile effettuare rapide operazioni
di controllo di appartenenza, oppure di differenza (ottenendo quindi l'elenco degli utenti paganti che non hanno
aggiornato all'ultima versione, in modo da poter per esempio inviare loro una email di sollecito).

Nei database relazionali, queste operazioni sono tipicamente possibile tramite delle query SQL, che il database
rende efficente tramite indici. Allo stesso modo, in questo esempio abbiamo utilizzato un insieme come indice 
esterno di una tabella, dovendolo però aggiornare manualmente. 

Questi sono i principali comandi che Redis prevede per gli insiemi, caratterizzati dal prefisso \verb|S|:

\begin{itemize}
	\medskip
	\item \textbf{SADD}. Aggiunge una stringa ad un insieme.
	\item \textbf{SREM}. Rimuove una stringa da un insieme.
	\item \textbf{SISMEMBER}. Controlla se una stringa è presente in un insieme.
	\item \textbf{SMEMBERS}. Elenca tutte le stringhe contenute nell'insieme.
	\item \textbf{SRANDMEMBER}. Restituisce una o più stringhe scelte casualmente tra quelle presenti.
	\item \textbf{SPOP}. Come ``SRANDMEMBER'', ma rimuove le stringhe scelte dall'insieme.
	\item \textbf{SUNION}. Restituisce l'unione tra due o più insiemi.
	\item \textbf{SINTER}. Restituisce l'interesezione tra due o più insiemi.
	\item \textbf{SDIFF}. Restituisce la differenza tra un insieme e uno o più altri insiemi.
	\item \textbf{SUNIONSTORE}, \textbf{SINTERSTORE}, \textbf{SDIFFSTORE}: Come i comandi precedenti, ma il risultato
	dell'operazione viene memorizzato in Redis in una nuova chiave specificata, e viene restituita solo la cardinalità
	del risultato.
\end{itemize}

Qui si può vedere una sessione interattiva di esempio di uso degli insiemi:

\medskip
\begin{lstlisting}
127.0.0.1:6379> SADD paying_users 88 37 4 66
(integer) 4
127.0.0.1:6379> SADD paying_users 37 111
(integer) 1
127.0.0.1:6379> SISMEMBER paying_users 111
(integer) 1
127.0.0.1:6379> SREM paying_users 37
(integer) 1
127.0.0.1:6379> SCARD paying_users
(integer) 4
127.0.0.1:6379> SISMEMBER paying_users 37
(integer) 0
127.0.0.1:6379> SADD updated_users 12 4 66
(integer) 3
127.0.0.1:6379> SDIFF paying_users updated_users
1) "88"
2) "111"
127.0.0.1:6379> SINTER paying_users updated_users
1) "4"
2) "66"
\end{lstlisting}


\subsection{Insiemi ordinati}

\subsection{HyperLogLog}

\section{Stored procedure in LUA}

(performance su comandi singoli, roundtrip di rete)


\section{Transazioni}


\section{Utilizzo come cache}

(TTL e timeout)


\section{Velocita}

(benchmark)


\section{Concorrenza}

Coerentemente con la sua archiettura orientata alla semplicità e al minimalismo, Redis ha un supporto primitivo
ma efficace per la concorrenza, che è basato sulla serializzazione delle operazioni. La scelta è dovuta al fatto che 
l'esecuzione di ciascun comando è (nella maggior parte dei casi) molto veloce anche con grossi dataset, come
si è visto, trattandosi di operazioni dirette su specifiche strutture dati completamente conservate in memoria, 
senza necessità di interrogare o aggioranare indici, né effettuare I/O di alcun tipo.

All'interno di ciasuna connessione TCP, il protocollo proibisce di eseguire comandi in parallelo, cioè il client 
deve necessariamente attendere il risultato di un comando prima di poter inviare il comando successivo (con l'unica
eccezione dell'esecuzione di una transazione, che però come abbiamo visto si configura di fatto come un unico
comando inviato a livello di protocollo).

Viceversa, per quanto concerne comandi inviati da più clienti (e quindi tramite più connessioni) in parallelo,
Redis provvede a serializzarli tra loro in automatico poiché il codice del server si basa su un singolo thread
di esecuzione, e gestisce l'accesso ai socket in parallelo tramite programmazione asincrona e accesso
I/O non bloccante, appoggiandosi a primitive quali ``select'', ``epoll'' o ``kqueue''. 

Questa architettura rende il codice di Redis più semplice e veloce, perché non è 
necessario acquisire né lock granulari nell'accesso ai singoli oggetti e strutture dati, né un lock globale per
serializzare gli accessi tra diversi thread in carico di gestire ciascuna connesione. L'uso di lock quale
per esempio un ``mutex'' implementato dalla libreria ``pthread'' avrebbe un impatto non solo a livello di semplicità di 
codice sorgente, ma anche di performance, poiché acquisire un lock impiega un tempo non trascurabile rispetto alla
reale esecuzione dell'operazioni richiesta, anche nel caso in cui non è conteso.

\section{Persistenza}








\chapter{I filtri di Bloom}

\section{Strutture dati e algoritmi probabilistici}

Le strutture dati e gli algoritmi che basano il proprio funzionamento sull'utilizzo di una componente
di casualità vengono detti ``probabilistici''. 

Ci sono due principali famiglie di algoritmi probabilistici:

\begin{itemize}
	\medskip

	\item Gli algoritmi detti ``Monte Carlo'', nei quali si effettua un campionamento casuale di
	valori in uno spazio di ricerca, consentendo il calcolo approssimato entro una certa
	probabilità. Il nome deriva dal Principato di Monte Carlo, conosciuto a livello mondiale come
	meta per il gioco d'azzardo. Gli algoritmi di questo tipo permettono solitamente di bilanciare
	il tempo di esecuzione con l'errore richiesto, poiché aumentando il numero di tentativi l'errore
	scende. Un esempio di questi algoritmi sono i test di primalità (cioè i test per verificare se
	un numero è primo) come il test di Fermat, nei quali si effettua controlli di una condizione
	necessaria ma non sufficiente e ci si convince che il numero è il primo quando con un
	determinato numero di tentativi.

	\item Gli algoritmi detti ``Las Vegas'', nei quali si utilizza la casualità durante l'algoritmo,
	ma viene prodotto sempre un risultato esatto, oppure l'algoritmo fallisce esplicitamente
	l'esecuzione. In questo caso dunque la componente casuale non è parte integrante della qualità
	risultato ottenuto, ma può influenzare il tempo d'esecuzione. Un esempio classico è l'algoritmo
	di ordinamento ``Quicksort'' in cui la scelta del pivot da utilizzare ad ogni passaggio è
	casuale, ma il risultato è sempre corretto. Il nome è stato proposto in \cite{lasvegas} per
	contrasto con Monte Carlo, dato che Las Vegas è un'altra meta conosciuta per il gioco d'azzardo.
\end{itemize}

Nelle strutture dati, invece, l'utilizzo del caso permette la memorizzazione di un'informazione
parziale, consentendo comunque di elaborarla almeno parzialmente ma con margini di errori
accettabili. Si tratta tipicamente di strutture dati molto specialistiche, utili ad implementare un
solo specifico algoritmo, poiché la perdita di informazione controllata è orientata a memorizzare il
minimo indispensabile per lo specifico scenario d'uso, ma impedendo ovviamente delle operazioni che
richiedono l'informazione completa, come anche la semplice enumerazione.

\section{Gli insiemi}

Un insieme è una struttura dati che memorizza un gruppo di elementi distinti, senza alcun ordine tra
essi (quindi né ordinamento intrinseco dei dati, né ordinamento esterno quale per esempio l'ordine
di inserimento).

Le operazioni principali che si possono effettuare sulla struttura dati sono le seguenti:

\begin{itemize}
	\medskip

	\item
	\textbf{Inserimento}: aggiunge un elemento all'insieme. Se l'elemento è già presente,
	l'operazione non modifica la struttura dati. In alcune implementazioni, l'operazione può 
	restituire un codice di errore per indicare che l'inserimento non è stato effettuato
	perché l'elemento era già presente.

	\item
	\textbf{Test di appartenenza}: controlla se un elemento è presente nell'insieme.

	\item
	\textbf{Cardinalità}: restituisce il numero di elementi presenti nell'insieme.

	\item
	\textbf{Unione}: restituisce un insieme che contiene tutti gli elementi presenti in almeno uno
	degli insiemi forniti in ingresso.

	\item
	\textbf{Intersezione}: restituisce un insieme che contiene tutti gli elementi presenti in tutti
	gli insiemi forniti in ingresso.
\end{itemize}

Alcune implementazioni offrono anche operazioni più avanzate legate alla teoria degli insiemi,
come per esempio la differenza o la differenza simmetrica. 

Un'implementazione classica di un'insieme prevede l'utilizzo di una tabella hash: l'inserimento
di un elemento avviene codificandolo con una funzione hash, memorizzandolo all'interno della
tabella, gestendo in modo opportuno le collisioni; il test di appartenenza può essere fatto così in
modo efficiente effettuando una ricerca nella tabella hash. Entrambe queste operazioni così 
implementate richiedono quindi un tempo d'esecuzione ammortizzato costante (\verb|O(k)|).

Al contrario di un array associativo, un insieme non ha dati associati all'elemento stesso, ma
nonostante questo è spesso possibile riutilizzare la stessa implementazione di tabella hash,
associando un dato nullo (non significativo) all'elemento. Questo accade comunemente nei linguaggi e
nelle librerie che mettono a disposizione una struttura dati di tipo array associativo (poiché più
completa); per esempio, il linguaggio Python fin dalla versione 1.0 prevedeva il ``dizionario'' come
struttura dati (un array associativo implementato con tabella hash), mentre l'``insieme'' è stato
aggiunto nella versione 2.3; prima di allora, era comune implementare l'insieme con un dizionario i
cui valori erano semplicemente valori booleani (di solito \verb|True|).

\section{Propriet\`a dei filtri di Bloom}

Un filtro di Bloom è una rappresentazione probabilistica di un insieme. Nel filtro infatti
non vengono memorizzati per intero tutti gli elementi, ma ne viene memorizzata solo una
rappresentazione parziale, che consente di effettuare comunque alcune operazioni, mantenendo però
dei margini di errore controllati. 

In particolare, il test di appartenenza restituisce un valore probabilistico, con la presenza di
falsi valori positivi (cioè il test restituisce un valore di successo anche per elementi non
effettivamente presenti). Il margine di errore può essere controllato tramite alcuni parametri, che
vengono stabiliti al momento della creazione del filtro stesso.

Vediamo nel dettaglio come i filtri si comportano relativamente alle operazioni principali sugli
insiemi elencate precedentemente:

\begin{itemize}
	\medskip

	\item
	\textbf{Inserimento}: è possibile inserire un elemento in un filtro, fino ad un numero massimo
	di elementi calcolabile in base ai parametri del filtro stesso. I filtri non possono essere
	infatti ridimensionati dopo la creazione e quando raggiungono il numero massimo di elementi,
	l'aggiunta di ulteriori elementi, seppure tecnicamente possibile, aumenta esponenzialmente 
	il margine di errore, rendendo di fatto il filtro inutilizzabile.

	Durante l'inserimento, è possibile effettuare contemporaneamente un test di appartenenza,
	restituendo una informazione probabilistica se l'elemento inserito era già presente o meno.

	\item
	\textbf{Test di appartenenza}: il filtro è in grado di verificare se un elemento è stato
	inserito con un certo margine di errore. In particolare, se il test restituisce un valore
	positivo, l'elemento è \emph{probabilmente presente}, con una soglia di errore controllato;
	ciò vuol dire che non si può avere la certezza che l'elemento sia effettivamente presente
	nell'insieme. Viceversa, se il test restituisce un valore negativo, si ha la certezza che il
	valore non è presente nell'insieme.

	\item
	\textbf{Cardinalità}: il filtro è in grado di effettuare una stima della cardinalità, compiendo
	un errore controllato sul valore restituito. I risultati rimangono consistenti anche al crescere
	della densità.

	\item
	\textbf{Unione}: è possibile calcolare con precisione l'unione di due filtri, purché siano
	creati con gli stessi parametri. Si noti che l'unione risultante dovrà rispettare il limite
	massimo di elementi previsto dai parametri di creazione, altrimenti non sarà utilizzabile.

	\item
	\textbf{Intersezione}: è possibile calcolare con precisione l'intersezione di due filtri,
	purché siano creati con gli stessi parametri.

\end{itemize}

\section{Scenari applicativi}

I filtri di Bloom hanno conosciuto negli anni una ampia adozione, in diversi scenari applicativi. 
In linea generale, consentono di ottenere una rappresentazione compatta di un insieme di dati,
che può essere usata sia come cache ad accesso veloce (in RAM) per un controllo preliminare
di esistenza di un elemento, sia trasferita in modo efficiente via rete ad un altro nodo per
fornire una vista parziale di un set di dati.

Un approfondimento degli scenari applicativi presentati qui può essere trovata in
\cite{bloomnetwork}.

\subsection{Controllo ortografico e sillabazione}

Quando furono introdotti nel 1970, Bloom stava lavorando ad un programma per la sillabazione del
testo inglese. Al contrario delle lingue latine, la sillabazione delle parole nella lingua inglese 
non è basata sui fonemi, ma sull'etimologia o la morfologia delle parole. Questo rende le regole
di sillabazione molto complesse, con numerose eccezioni. Bloom pensò di ottimizzare il programma
utilizzando un filtro per memorizzare tutte le parole che richiedevano un'eccezione; il programma
poteva così verificare velocemente se una parola era soggetta ad eccezione o meno con un test
di appartenenza nel filtro; nel caso il test restituisse un valore positivo, il programma poteva
effettuare un controllo più lento nel dizionario completo delle eccezione. Le parole soggette
a falso positivo venivano riportate al caso standard al termine del controllo nel dizionario.

Allo stesso modo, i filtri di Bloom vennero usati per i primi programmi di controllo ortografico
negli anni 80. Era possibile infatti memorizzare una versione compatta di un dizionario di tutte le
parole conosciute (complete anche tutte le forme verbali che possono essere costruite tramite
algoritmo), ed utilizzarlo per effettuare il controllo; in questo caso, i falsi positivi
corrispondevano a parole non esistenti che veniva erroneamente ignorate durante il controllo. Visto
che all'epoca non era possibile distribuire l'intero dizionario, questa soluzione era un compromesso
accettabile.

\subsection{Proxy Cache distribuita}

In \cite{bloomproxy}, viene introdotto uno dei primi casi di utilizzo dei filtri di Bloom per
ottimizzare i trasferimenti di rete. Lo scenario prevedeva una rete di proxy web, che comunicavano
tra loro. Nel momento in cui veniva effettuata una richiesta ad un proxy per una pagina web non
presente nella cache, il proxy poteva comunicare con gli altri proxy della rete, chiedendo loro
se qualcuno aveva nella propria cache la pagina richiesta, in modo da accedervi senza contattare
il server d'origine.

Per diminuire il traffico generato per la comunicazione tra proxy, fu introdotto un meccanismo
tramite il quale ciascun proxy periodicamente pubblicava in broadcast un filtro di Bloom contenente
l'elenco dei siti memorizzati nella propria cache. Gli altri proxy memorizzavano il filtro e
potevano usarlo per verificare se un sito era presente in cache senza causare traffico di rete. I
falsi positivi corrispondevano dunque a richieste di rete potenzialmente inutili, ma in ogni caso
una piccola frazione rispetto alla quantità di traffico generato in precedenza per ogni richiesta.


\subsection{Google SafeBrowsing}

Il browser Chrome di Google utilizzava (fino al 2012) i filtri di Bloom per implementare
efficientemente la funzionalità SafeBrowsing \cite{safebrowsing}, che protegge gli utenti bloccando
automaticamente l'accesso ai siti contenenti \emph{malware} o \emph{phishing}.

Google, infatti, mantiene sui propri server un enorme elenco di tipo \emph{blacklist} di indirizzi
web (URL) considerate pericolose per gli utenti. L'elenco è in continua evoluzione: nuovi indirizzi
vengono infatti aggiunti e rimossi con alta frequenza ogni ora.

Un applicativo che volesse verificare la presenza di un sito all'interno dell'elenco potrebbe
effettuare una interrogazione via API ai server di Google; il tempo di esecuzione di questo
controllo comprende però almeno 1 \emph{roundtrip} di rete tra il client e il server, che può
richiedere tempi variabile a seconda della qualità della connessione Internet, ma può salire fino a
centinaia di millisecondi per una connessione mobile (anche in presenza di segnale ottimale). Questo
genere di attesa prima della visualizzazione di una pagina è considerato non accettabile per un
browser come Google Chrome, il quale cerca di minimizzare il tempo di apertura dei siti.

Scaricare l'intera lista di siti richiederebbe probabilmente molta banda, soprattutto per le
numerose sincronizzazione che sarebbero necessarie periodicamente. Vista l'alta volatilità della
lista, infatti, sarebbe necessario effettuare sincronizzazioni con alta frequenza, per evitare di
consultare una lista troppo vecchia, e questi aggiornamenti avrebbero impatti importanti
sull'utilizzo di banda Internet da parte del client.

La soluzione che fu scelta inizialmente fu quella di affidarsi ad un filtro di Bloom: i server
di Google infatti costruivano un filtro contenente tutte le URL presenti nella blacklist, e
trasferivano il filtro al client. Grazie alle proprietà dei filtri di Bloom, i dati da trasferire
erano molto compatti, e potevano essere conservati in RAM anche sui dispositivi mobili.

Ogni volta che l'utente navigava su un nuovo sito, il client verificava prima se l'indirizzo era
presente nel filtro: nel caso più comune (sito non malevole), il filtro restituiva correttamente
esito negativo, e il browser consentiva l'accesso all'utente senza nessun rallentamento. Solo nei
casi in cui il filtro restituiva esito positivo, il client effettuava una chiamata alle API di
Google per verificare se l'indirizzo era veramente malevolo o se si trattasse di un falso positivo;
con parametri di creazione adeguati, era possibile fare in modo che questa chiamata avvenisse
pochissime volte nel corso di una navigazione normale, impattando così il meno possibile i tempi
di apertura dei siti.

Con il passare del tempo, la lista è cresciuta troppo, e di conseguenza il filtro ha iniziato
ad avere dimensioni importanti; di conseguenza, è stato scelto di usare un'altra struttura dati
probabilistica (gli insiemi di Golomb-Rice \cite{golomb}), che ha una rappresentazione in memoria
compressa, dunque più compatta, sebbene più lenta da utilizzare \cite{golomb-safebrowsing}. 

\subsection{Riconciliazione di una rete di distribuzione contenuti}

Le reti di distribuzione contenuti (\emph{Content Delivery Network} o \emph{CDN}) sono delle reti
di server distribuiti geograficamente che memorizzano copie di dati statici (immagini, font, 
fogli di stile, sorgenti javascript) richiesti durante la navigazione di siti ad alto traffico. 
Memorizzando una copia di questi contenuti sui nodi della CDN, il contenuto può essere offerto
al visitatore da un server geograficamente più vicino a lui e quindi con una latenza di rete
inferiore, migliorando la velocità di accesso al sito; inoltre, il traffico si distribuisce
così su tanti nodi, diminuendo l'impatto del traffico sul server principale.

In alcuni casi, può essere richiesto che i nodi della CDN si \emph{riscaldino}, popolando le
proprie cache di contenuti che si ritiene verranno richiesti a breve dagli utenti. In questo caso,
le CDN lavorano con una logica di rete P2P, trasferendo i dati da nodo a nodo. Per sincronizzare
il contenuto del nodo $N1$ con il nodo $N2$, $N1$ invia a $N2$ un filtro di Bloom contentente
l'elenco dei contenuti presenti nella propria cache. $N2$ può quindi utilizzare il filtro
per individuare i contenuti che $N1$ non possiede, ed inviargieli. In caso di falso positivo,
un dato risulterà erroneamente posseduto da $N1$ e non verrà quindi inviato; trattandosi di una
operazione di ottimizzazione del traffico, è sufficiente che non avvenga con frequenza troppo 
elevata per non impattare sulle prestazioni: infatti $N1$ non avrà il dato disponibile all'arrivo
della prima richiesta, e si occuperà di scaricarlo in quel momento. 

L'azienda Akamai (leader di mercato nel settore delle CDN) utilizza i filtri di Bloom anche per
evitare di memorizzare contenuti richiesti una sola volta (come descritto in \cite{bloomakamai}).
Infatti ogni nodo conserva un filtro di Bloom dei contenuti richiesti almeno una volta. Alla prima
richiesta, il contenuto non è ancora presente nel filtro e viene aggiunto ma non viene cachato
dal nodo. In questo modo, si evita di inserire in cache contenuti non popolari che possono 
utilizzare spazio meglio utilizzato da contenuti più richiesti. I falsi positivi corrispondono
a contenuti richiesti la prima volta che risultano però già richiesti; in questo caso, poiché
la probabilità è bassa, l'ottimizzazione ha comunque successo nel non appesantire troppo la cache.

\subsection{Routing ottimale in reti P2P}

All'interno delle reti puramente P2P, il routing è l'algoritmo che consente ad ogni nodo di
decidere verso quale altro nodo inviare un pacchetto. È possibile pensare alla rete P2P come
ad un grafo, e al problema del routing come alla ricerca di un percorso tra nodi all'interno
del grafo. La ricerca del percorso ottimale è di solito possibile tramite dei protocolli di 
\emph{path discovery}, che sono però abbastanza onerosi. Di conseguenza, è possibile memorizzare
l'esito della ricerca del percorso all'interno di un bloom filter associato ad ogni arco uscente.
Se viene richiesto di effettuare lo stesso routing, il bloom filter consente di individuare
rapidamente l'arco da utilizzare per inviare il pacchetto. I falsi positivi in questo caso
corrispondono a routing non ottimali: il pacchetto viene inviato erroneamente al nodo sbagliato,
ma probabilmente raggiungerà comunque poi la destinazione, sebbene non nel modo più efficiente. 

Le reti P2P sono per loro stessa natura dinamiche: i nodi si connettono e sconnettono con una certa
frequenza. Quando un nodo si disconnette, i nodi collegati possono decidere di buttare via il
filtro associato all'arco che scompare, o in alternativo unirlo ad un altro filtro di un nodo
magari ad alto traffico con molta banda a disposizione (utilizzando l'operazione di unione
dei filtri).

\section{Funzionamento dei filtri}

Un filtro di Bloom è un array di bit, utilizzato come tabella hash, di dimensione arbitraria
\verb|M| fissata al momento della creazione in base ai parametri (vedi \ref{sec:bloomparms}).
Quando il filtro è vuoto, tutti i bit sono impostati a 0. La seguente tabella mostra un filtro
vuoto composto di 11 bit:

\begin{center}
  \begin{tabular}{*{11}{|c}|}
  	\multicolumn{1}{c}{0} & \multicolumn{1}{c}{1} & \multicolumn{1}{c}{2} &
  	\multicolumn{1}{c}{3} & \multicolumn{1}{c}{4} & \multicolumn{1}{c}{5} &
  	\multicolumn{1}{c}{6} & \multicolumn{1}{c}{7} & \multicolumn{1}{c}{8} &
  	\multicolumn{1}{c}{9} & \multicolumn{1}{c}{10} \\
    \hline
    0 & 0 & 0 & 0 & 0 & 0 & 0 & 0 & 0 & 0 & 0 \\
    \hline
  \end{tabular}
\end{center}

\subsection{Inserimento di un elemento}

L'inserimento di un elemento $X$ avviene calcolando alcune funzioni di hash $h_i(X)$ prestabilite
sull'elemento (il numero delle funzioni, chiamato $K$, è anch'esso un parametro del filtro). I $K$
valori risultanti $h_0(X), h_1(X), ... , h_K(X)$ sono utilizzati per identificare singoli bit
all'interno dell'array (per esempio, scegliendo il bit con indice $h_i(X) \bmod M$), e i bit così
identificati vengono impostati a 1. Si noti che nessuna altra informazione sull'elemento viene
memorizzata dentro il filtro.

Continuando l'esempio seguente, supponiamo di utilizzare $K=2$ funzioni di hash $h_1(x)$ e $h_2(x)$
e di voler inserire 3 elementi $\{ A, B, C \}$ nel filtro. La seguente tabella mostra i valori
delle funzioni di hash per ciascun elemento:

\begin{center}
	\begin{tabular}{ l c c c }
		 & A & B & C \\
		\hline
		$h_1(x)$ & 1 & 10 & 7 \\
		$h_2(x)$ & 8 & 4 & 1 \\	
		\hline
	\end{tabular}
\end{center}

Per inserire l'elemento $A$, è dunque sufficiente impostare ad 1 i bit $h_1(A) = 1$ e $h_2(A) = 8$:

\begin{center}
  \begin{tabular}{*{11}{|c}|}
  	\multicolumn{1}{c}{0} & \multicolumn{1}{c}{1} & \multicolumn{1}{c}{2} &
  	\multicolumn{1}{c}{3} & \multicolumn{1}{c}{4} & \multicolumn{1}{c}{5} &
  	\multicolumn{1}{c}{6} & \multicolumn{1}{c}{7} & \multicolumn{1}{c}{8} &
  	\multicolumn{1}{c}{9} & \multicolumn{1}{c}{10} \\
    \hline
    0 & \cellcolor{blue!25}1 & 0 & 0 & 0 & 0 & 0 & 0 & \cellcolor{blue!25}1 & 0 & 0 \\
    \hline
  \end{tabular}
\end{center}

Per inserire l'elemento $B$, impostiamo a 1 i bit $10$ e $4$:

\begin{center}
  \begin{tabular}{*{11}{|c}|}
  	\multicolumn{1}{c}{0} & \multicolumn{1}{c}{1} & \multicolumn{1}{c}{2} &
  	\multicolumn{1}{c}{3} & \multicolumn{1}{c}{4} & \multicolumn{1}{c}{5} &
  	\multicolumn{1}{c}{6} & \multicolumn{1}{c}{7} & \multicolumn{1}{c}{8} &
  	\multicolumn{1}{c}{9} & \multicolumn{1}{c}{10} \\
    \hline
    0 & 1 & 0 & 0 & \cellcolor{blue!25}1 & 0 & 0 & 0 & 1 & 0 & \cellcolor{blue!25}1 \\
    \hline
  \end{tabular}
\end{center}

Per inserire l'elemento $C$, impostiamo a 1 i bit $7$ e $1$. Si noti che il bit $1$ era già
stato impostato dall'elemento $A$, ma questo non causa problemi: si tratta di un normale 
conflitto di funzione di hash ($h_1(A) = h_2(C)$) che è possibile ignorare:

\begin{center}
  \begin{tabular}{*{11}{|c}|}
  	\multicolumn{1}{c}{0} & \multicolumn{1}{c}{1} & \multicolumn{1}{c}{2} &
  	\multicolumn{1}{c}{3} & \multicolumn{1}{c}{4} & \multicolumn{1}{c}{5} &
  	\multicolumn{1}{c}{6} & \multicolumn{1}{c}{7} & \multicolumn{1}{c}{8} &
  	\multicolumn{1}{c}{9} & \multicolumn{1}{c}{10} \\
    \hline
    0 & \cellcolor{blue!5}1 & 0 & 0 & 1 & 0 & 0 & \cellcolor{blue!25}1 & 1 & 0 & 1 \\
    \hline
  \end{tabular}
\end{center}

A questo punto, l'inserimento dei 3 elementi è terminato.

\subsection{Test di appartenenza}

Il test di appartenenza di un elemento segue un processo simmetrico: vengono eseguite le $K$
funzioni di hash e si identificano così $K$ bit dell'array di cui verificare lo stato. Se tutti i 
bit sono impostati a 1, l'elemento è \emph{probabilmente presente} all'interno del filtro; se almeno 
uno dei bit è impostato a 0, l'elemento è \emph{sicuramente assente} all'interno del filtro.

Il motivo per cui il test di appartenenza non può asserire con certezza la presenza di un elemento
è dovuto all'assenza di un meccanismo di rilevamento delle collisioni delle funzioni di hash tra
elementi distinti; quando il test verifica se un determinato bit è impostato a 1, non può sapere
se quel bit è stato impostato tramite l'inserimento dell'elemento su cui si sta effettuando il test,
o tramite l'inserimento di un altro elemento che è in conflitto a livello di funzione di hash. Se
un elemento che non è presente utilizza, per il test di primalità, $K$ bit che sono stati
precedenti accesi da altri elementi tramite le varie funzioni di hash ad essi applicate, l'elemento
risulterà falsamente presente nel filtro, causando un cosiddetto \emph{falso positivo}.  

Continuando l'esempio del filtro visto in precedenza, supponiamo di voler effettuare il test di 
appartenenza sui seguenti elementi:

\begin{center}
	\begin{tabular}{ l c c c }
		 & B & E & F \\
		\hline
		$h_1(x)$ & 10 & 0 & 4 \\
		$h_2(x)$ & 4 & 7 & 8 \\	
		\hline
	\end{tabular}
\end{center}

Per verificare la presenza dell'elemento $B$, verifichiamo se i bit $10$ e $4$ sono impostati a 1;
poiché l'elemento è stato effettivamente inserito in precedenza, questa condizione è banalmente
vera:

\begin{center}
  \begin{tabular}{*{11}{|c}|}
  	\multicolumn{1}{c}{0} & \multicolumn{1}{c}{1} & \multicolumn{1}{c}{2} &
  	\multicolumn{1}{c}{3} & \multicolumn{1}{c}{4} & \multicolumn{1}{c}{5} &
  	\multicolumn{1}{c}{6} & \multicolumn{1}{c}{7} & \multicolumn{1}{c}{8} &
  	\multicolumn{1}{c}{9} & \multicolumn{1}{c}{10} \\
    \hline
    0 & 1 & 0 & 0 & \cellcolor{green!35}1 & 0 & 0 & 1 & 1 & 0 & \cellcolor{green!35}1 \\
    \hline
  \end{tabular}
\end{center}

La verifica dell'elemento $E$ comprende il test dei bit $0$ e $7$. Si noti che il bit $7$ era
stato precedentemente impostato durante l'inserimento dell'elemento $C$ ($h_1(C) = h_2(E)$), ma
il test fallisce (restituendo il risultato corretto) poiché il bit $0$ è impostato a 0, e il test
prevede che tutti i bit controllati debbano essere impostati a 1:

\begin{center}
  \begin{tabular}{*{11}{|c}|}
  	\multicolumn{1}{c}{0} & \multicolumn{1}{c}{1} & \multicolumn{1}{c}{2} &
  	\multicolumn{1}{c}{3} & \multicolumn{1}{c}{4} & \multicolumn{1}{c}{5} &
  	\multicolumn{1}{c}{6} & \multicolumn{1}{c}{7} & \multicolumn{1}{c}{8} &
  	\multicolumn{1}{c}{9} & \multicolumn{1}{c}{10} \\
    \hline
    \cellcolor{red!35}0 & 1 & 0 & 0 & 1 & 0 & 0 & \cellcolor{green!35}1 & 1 & 0 & 1 \\
    \hline
  \end{tabular}
\end{center}

La verifica dell'elemento $E$ comprende il test dei bit $4$ e $8$. Entrambi i bit erano stati
impostati in precedenza durante l'inserimento di altri elementi ($h_1(F) = h_2(E)$, $h_2(F) =
h_2(A)$), e il test dunque restituisce vero. Si tratta di un caso di \emph{falso positivo}, poiché
l'elemento $E$ non era stato precedentemente inserito nel filtro.

\begin{center}
  \begin{tabular}{*{11}{|c}|}
  	\multicolumn{1}{c}{0} & \multicolumn{1}{c}{1} & \multicolumn{1}{c}{2} &
  	\multicolumn{1}{c}{3} & \multicolumn{1}{c}{4} & \multicolumn{1}{c}{5} &
  	\multicolumn{1}{c}{6} & \multicolumn{1}{c}{7} & \multicolumn{1}{c}{8} &
  	\multicolumn{1}{c}{9} & \multicolumn{1}{c}{10} \\
    \hline
    0 & 1 & 0 & 0 & \cellcolor{green!35}1 & 0 & 0 & 1 & \cellcolor{green!35}1 & 0 & 1 \\
    \hline
  \end{tabular}
\end{center}


\subsection{Scelta delle funzioni di hash}

Utilizzo del double-hashing come metodo efficiente
Utilizzo del partizionamento in sezioni

Dillinger, Peter C.; Manolios, Panagiotis (2004b), "Bloom Filters in Probabilistic Verification", Proceedings of the 5th International Conference on Formal Methods in Computer-Aided Design, Springer-Verlag, Lecture Notes in Computer Science 3312


\subsection{Stima della cardinalità}
\label{sec:bloomcard}

È possibile effettuare una stima della cardinalità \verb|N| di un filtro di Bloom, derivandola
dalla densità, come già mostrato in \cite{bloomcard} \cite{bloomscalable}.

Si noti che un filtro di dimensione $M$ con $K$ funzioni di hash può essere implementato, senza
perdita di genericità, con $K$ array distinti di $m=\frac{M}{K}$ bit ciascuno, chiamati
\emph{sezioni}; in questo caso, le funzioni di hash dovranno essere utilizzate per indicizzare
ciascuno un bit all'interno di una sezione distinta, utilizzando per esempio $h_i(X) \bmod m$.

Ciascuna sezione inizia il ciclo di vita con tutti i bit impostati a 0. Dopo l'inserimento di un
elemento nel filtro, un solo bit sarà stato impostati a 1, per cui la probabilità $p$ che un bit
della sezione sia impostato a 1 (detta anche \emph{densità}) è $\frac{1}{m}$, mentre la probabilità
$q$ che un bit sia impostato a 0 è $ q=1-\frac{1}{m}$. Dopo $n$ inserimenti, le probabilità saranno
quindi: 

$$ q = (1-\frac{1}{m})^n $$
$$ p = 1-(1-\frac{1}{m})^n $$

Per ricavare $n$, si ricordi la seguente serie di Taylor:

$$ e^{-\frac{1}{x}} = 1 - \frac{1}{x} + \mathcal{O}((\frac{1}{x})^2) $$

dal quale si deriva:

$$ p \approx 1-(e^{-\frac{1}{m}})^n $$
$$ \ln(1-p) \approx -\frac{n}{m} $$
$$ n \approx -m\ln(1-p) $$
$$ n \approx -\frac{M}{K}\ln(1-p) $$

che è possibile valutare quindi estraendo la densità $p$ del filtro da un semplice conteggio
dei bit impostati ad 1 in un dato momento.

I grafici nella figura \ref{fig:bloomerror} mostrano l'errore di calcolo della cardinalità
effettuato in un filtro al crescere della densità, e al numero di funzioni di hash. Come si può
vedere, l'errore effettuato dalla stima non diverge anche per valori di densità elevati.

\begin{figure}
	\centering

	\begin{subfigure}[t]{280px}
	\centering\includegraphics[width=280px]{bloom_card_error_1h}
	\subcaption{M=8192, K=1}
	\end{subfigure}

	\begin{subfigure}[t]{280px}
	\centering\includegraphics[width=280px]{bloom_card_error_2h}
	\subcaption{M=8192, K=2}
	\end{subfigure}

	\begin{subfigure}[t]{280px}
	\centering\includegraphics[width=280px]{bloom_card_error_3h}
	\subcaption{M=8192, K=3}
	\end{subfigure}

	\caption{Errore sulla cardinalità nei filtri di Bloom}
	\label{fig:bloomerror}
\end{figure}

\subsection{Unione ed intersezione}

Dati due filtri creati con gli stessi parametri ed utilizzanti le stesse funzioni di hash, è
possibile calcolare l'unione semplicemente applicando l'operazione di \verb|OR| bit a bit agli
array. Questa operazione non causa alcuna perdita di informazione né pessimizzazione, poiché l'array
risultante è il medesimo che si sarebbe ottenuto aggiungendo in un solo filtro tutti gli elementi
presenti nel primo e nel secondo filtro in input.

È possibile in modo simile calcolare l'intersezione tra due filtri applicando l'operazione di
\verb|AND| bit a bit agli array. In questo caso però il risultato non è ottimale: l'array risultante
potrebbe contenere alcuni bit impostati a 1 che non risulterebbero tali se si fosse creato
direttamente un filtro aggiungendo solo gli elementi facenti parte dell'intersezione, e di
conseguenza la probabilità di falsi positivi risultante dal filtro intersezione è potenzialmente non
ottimale.

Supponiamo per esempio di avere due filtri che contengono rispettivamente gli elementi ${ A, B }$
e ${ A, C }$. Le funzioni di hash sugli elementi sono le seguenti:

\begin{center}
	\begin{tabular}{ l c c c }
		 & A & B & C \\
		\hline
		$h_1(x)$ & 5 & 2 & 7 \\
		$h_2(x)$ & 9 & 6 & 2 \\	
		\hline
	\end{tabular}
\end{center}

I filtri risultanti saranno i seguenti:

\begin{center}
  \begin{tabular}{*{11}{|c}|}
  	\multicolumn{1}{c}{0} & \multicolumn{1}{c}{1} & \multicolumn{1}{c}{2} &
  	\multicolumn{1}{c}{3} & \multicolumn{1}{c}{4} & \multicolumn{1}{c}{5} &
  	\multicolumn{1}{c}{6} & \multicolumn{1}{c}{7} & \multicolumn{1}{c}{8} &
  	\multicolumn{1}{c}{9} & \multicolumn{1}{c}{10} \\
    \hline
    0 & 0 & B & 0 & 0 & A & B & 0 & 0 & A & 0 \\
    \hline
  \end{tabular}
\end{center}

\begin{center}
  \begin{tabular}{*{11}{|c}|}
  	\multicolumn{1}{c}{0} & \multicolumn{1}{c}{1} & \multicolumn{1}{c}{2} &
  	\multicolumn{1}{c}{3} & \multicolumn{1}{c}{4} & \multicolumn{1}{c}{5} &
  	\multicolumn{1}{c}{6} & \multicolumn{1}{c}{7} & \multicolumn{1}{c}{8} &
  	\multicolumn{1}{c}{9} & \multicolumn{1}{c}{10} \\
    \hline
    0 & 0 & C & 0 & 0 & A & 0 & C & 0 & A & 0 \\
    \hline
  \end{tabular}
\end{center}

dove ogni cella impostata a 1 viene identificata con una lettera che si riferisce all'elemento che
l'ha impostata durante l'inserimento. 

L'insieme unione è $\{ A, B \} \cup \{ A , C \} = \{ A, B, C \}$, e il filtro risultante
dall'operazione di \verb|OR| bit a bit, come si può vedere, è ottimale poiché è lo stesso che
avremmo ottenuto per inserimento diretto:

\begin{center}
  \begin{tabular}{*{11}{|c}|}
  	\multicolumn{1}{c}{0} & \multicolumn{1}{c}{1} & \multicolumn{1}{c}{2} &
  	\multicolumn{1}{c}{3} & \multicolumn{1}{c}{4} & \multicolumn{1}{c}{5} &
  	\multicolumn{1}{c}{6} & \multicolumn{1}{c}{7} & \multicolumn{1}{c}{8} &
  	\multicolumn{1}{c}{9} & \multicolumn{1}{c}{10} \\
    \hline
    0 & 0 & B,C & 0 & 0 & A & B & C & 0 & A & 0 \\
    \hline
  \end{tabular}
\end{center}

L'insieme intersezione è ${ A, B } \cap { A , C } = { A }$, ma il filtro risultante dall'operazione
di \verb|AND| bit a bit è il seguente:

\begin{center}
  \begin{tabular}{*{11}{|c}|}
  	\multicolumn{1}{c}{0} & \multicolumn{1}{c}{1} & \multicolumn{1}{c}{2} &
  	\multicolumn{1}{c}{3} & \multicolumn{1}{c}{4} & \multicolumn{1}{c}{5} &
  	\multicolumn{1}{c}{6} & \multicolumn{1}{c}{7} & \multicolumn{1}{c}{8} &
  	\multicolumn{1}{c}{9} & \multicolumn{1}{c}{10} \\
    \hline
    0 & 0 & B,C & 0 & 0 & A & 0 & 0 & 0 & A & 0 \\
    \hline
  \end{tabular}
\end{center}

Il bit 2 rimane acceso a causa di un conflitto tra le funzioni di hash ($h_1(B) = h_2(C)$), ma
né l'elemento $B$ né l'elemento $C$ sono presenti nell'insieme intersezione. Se il filtro fosse
stato creato da zero, inserendo solamente $A$, il bit 2 risulterebbe impostato a zero.

\section{Ottimizzazione dei parametri}
\label{sec:bloomparms}

Ci sono diversi parametri di interesse quando si analizza un filtro di Bloom, riassunti in questa
tabella.

\medskip
\begin{tabular}{ |l|l| }
  \hline
  \multicolumn{2}{|c|}{Parametri} \\
  \hline
  $M$ & Dimensione in bit dell'array \\
  $K$ & Numero delle funzioni di hash (e delle sezioni) \\
  $m$ & Dimensione in bit di una sezione \\
  $N$ & Numero di elementi presenti nel filtro \\
  $p$ & Densità \\
  $P$ & Probabilità di un falso positivo \\
  \hline
\end{tabular}

\medskip

Vediamo ora come è possibile ottimizzare i parametri, cercando di minimizzare $P$, la probabilità
di un falso positivo, come descritto in \cite{bloomscalable}.

La probabilità di trovare un bit impostato a 1 in una sezione è $p$, di conseguenza la probabilità 
di un falso positivo, cioè di trovare i bit di tutte le sezioni già impostati, è:

$$ P = p^K $$

Ricordando che $M=km$, otteniamo:

$$ m = M\frac{\ln(p)}{\ln(P)} $$

e sostituendo nella formula della cardinalità, otteniamo:

$$ N \approx -M\frac{\ln(p)\ln(1-p)}{\ln(P)} $$

Supponiamo ora di fissare $M$ (dimensione del filtro) e $P$ (probabilità di un falso positivo), e di
voler cercare qual è la densità $p$ che massimizza il numero di elementi presenti nell'array. È
facile convincersi che il massimo della funzione $\ln(p)\ln(1-p)$ si trova con $p=0.5$, risultato
che ci indica che il punto di utilizzo ottimo di un filtro di Bloom è quando la densità è
esattamente al 50\%, cioè metà dei bit è impostato a 1.

Sostituendo quindi $p=0.5$ come valore ottimo, otteniamo:

$$ N \approx -M\frac{\ln(0.5)^2}{\ln(P)} $$
$$ N \approx \ceil*{M\frac{0.480453014}{|\ln(P)|}} $$

che ci permette di calcolare il numero di elementi memorizzabili in un filtro prima di raggiungere
la densità di 0.5 e dunque la probabilità $P$ di falsi positivi richiesta. Oppure, ricavando
le altre variabili:

$$ M = \ceil*{n\frac{|\ln(P)|}{0.480453014}} $$
$$ P = e^{-\ln(0.5)^2\frac{M}{N}} = 0.5^{ln(2)\frac{M}{N}} \approx 0.6185^{\frac{M}{N}} $$

possiamo calcolare la dimensione ottimale di un filtro dato il numero di elementi che si vuole
memorizzare e la probabilità di falsi positivi attesa, oppure ancora i falsi positivi attesi
data la dimensione di un filtro e il numero di elementi inseriti. 

Inoltre, da $P = p^k$, sostituendo nuovamente $p=0.5$, possiamo ricavare:

$$ K = \ceil*{log_2{\frac{1}{P}}} = \ceil*{-log_2{P}} $$

con cui calcolare il numero ottimale di funzioni di hash per un data probabilità attesa di falso
positivo.

Vediamo ora due esempi di derivazione di parametri ottimali. 

\begin{itemize}
	\medskip

	\item Supponiamo di voler memorizzare $N=25000$ elementi in un filtro con una probabilità di
	falso positivo $P=0.5\%$; possiamo ricavare i parametri necessari in questo modo:

	$$ M = \ceil*{n\frac{|\ln(P)|}{0.480453014}} = \ceil*{25000\frac{|ln(0.005)|}{0.480453014}} = 275694 $$
	$$ K = \ceil*{-log_2{P}} = \ceil*{-log_2{0.005}} = 8 $$

	Quindi è necessario utilizzare un filtro di circa 33 KiB con 8 funzioni di hash.

	\item Supponiamo di avere 1 MiB di RAM a disposizione per creare un filtro di Bloom che deve
	memorizzare 2 milioni di elementi. Qual è il valore minimo di falso positivo che possiamo
	ottenere, e con quante funzioni di hash lo otterremo? È presto detto:

	$$ M = 1024 x 1024 x 8 = 8388608 $$
	$$ N = 2\,000\,000 $$
	$$ P = 0.6185 ^ \frac{M}{N} = 14.63\% $$
	$$ K = \ceil*{-log_2{P}} = 3 $$

	Quindi possiamo arrivare ad avere circa il 14\% di falsi positivi utilizzando 3 funzioni di
	hash.
\end{itemize}

\begin{figure}
	\centering

	\begin{subfigure}[t]{280px}
	\centering\includegraphics[width=280px]{bloom_parms_1}
	\subcaption{Numero di elementi in base alla probabilità attesa}
	\end{subfigure}

	\begin{subfigure}[t]{280px}
	\centering\includegraphics[width=280px]{bloom_parms_2}
	\subcaption{Probabilità attesa in base al numero di elementi}
	\end{subfigure}

	\caption{Andamento dei parametri ottimale in un filtro di 64 MiB}
\end{figure}


\section{Varianti dei filtri di Bloom}














\chapter{Aggiunta dei filtri di Bloom a Redis}

\section{Obiettivo}

L'obiettivo della modifica a Redis svolta per questa tesi è introdurre il supporto per i filtri di
Bloom. È stato effettuata prima una fase di progettazione dei nuovi comandi da aggiungere e di
conseguenza della variante esatta di filtro da implementare; è stata preparata una prima
implementazione completa di copertura tramite test automatizzati; infine, sono state effettuate
delle analisi sui parametri di configurazione di default, cercando di studiare il compromesso tra
utilizzo della memoria, tempo di esecuzione, e probabilità di falsi positivi.

\section{Scelta della variante di filtro}

Nella fase iniziale di design, abbiamo comunicato più volte con l'autore di Redis (Salvatore
Sanfilippo), cercando di stabilire insieme i requisiti che l'implementazione del filtro dovesse
avere perché la modifica corrispondesse agli standard di Redis.

In particolare, abbiamo identificato i seguenti requisiti:

\begin{itemize}
	\medskip
	\item \textbf{Dinamicità}: tutte le strutture dati di Redis sono dinamiche, cioè è possibile
	aggiungere elementi illimitati, con l'unico vincolo rappresentato dalla memoria a disposizione
	sul sistema. Nessuna struttura dati richiede un dimensionamento fisso.

	\item \textbf{Scalabilità}: l'utilizzo della memoria deve essere molto contenuto quando la
	struttura dati contiene pochi elementi, e crescere gradualmente all'aumentare degli elementi.
	Eventuali preallocazioni devono essere molto limitate. Per esempio, non sarebbe accettabile che
	il filtro possa occupare \SI{1}{\mega\byte} di RAM alla creazione.

	\item \textbf{Scarsa configurabilità}: Redis cerca di offrire strutture dati che siano
	preconfigurate con default accettabili per la maggior parte dei casi d'uso, e offre la minima
	configurabilità necessaria ad operare negli scenari applicativi previsti. In particolare,
	i filtri di Bloom hanno diversi parametri a disposizione ed è quindi richiesto delle scelte
	oculate.
\end{itemize}

Per soddisfare i requisiti, ci siamo orientati su implementare i {filtri di Bloom
scalabili}~\ref{sec:bloomscalable}, i quali eliminano la necessità di determinare in anticipo il
dimensionamento della struttura (in termini di byte o di numero di elementi), e la cui crescita
esponenziale di occupazione della memoria consente una sostanziale scalabilità.

Abbiamo ritenuto essenziale esporre il parametro $P$ di probabilità di falso positivo poiché
strettamente legato al tipo di scenario applicativo, ma nascondendo all'utente tutti i parametri più
tecnici (fattore di intensificazione $r$, fattore di crescita $s$, tipologia di funzione di hash).
Anche per il parametro $P$ ci siamo orientati su offrire un default, rendendo più semplice per
l'utente iniziare l'utilizzo della struttura dati, anche in fase di sviluppo.

\section{Progettazione dei nuovi comandi}

Nel progettare i nuovi comandi, abbiamo cercato di adeguarci il più possibile allo stile dei
comandi esistenti, cercando di mantenere le proprietà evinte durante la fase di analisi.

Abbiamo scelto il prefisso \verb|BF| per la famiglia di comandi, indicando chiaramente la
struttura dati che viene manipolata.

Questi sono i comandi implementati:

\begin{itemize}
	\medskip
	\item \textbf{BFADD <key> [ERROR p] ELEMENTS <ele1> <ele2> ...}: aggiunge uno o più elementi al
	filtro di Bloom specificato, creandolo se non già esistente. Restituisce il numero degli
	elementi che risultavano non presenti al momento dell'inserimento (dato soggetto quindi ad
	errore).

	L'errore $P$ può essere indicato opzionalmente solo al momento della creazione, cioè del primo
	inserimento; se viene specificato in un filtro già esistente, e il valore specificato è diverso
	da quello con cui il filtro è stato creato, il comando restituisce un errore senza effettuare
	alcun inserimento.

	\item \textbf{BFEXIST <key> <ele>}: Verifica la presenza di un elemento nel filtro specificato
	(test di appartenenza). Se il filtro non esiste, il comando restituisce falso.

	\item \textbf{BFCOUNT <key>}: Restituisce una stima del numero di elementi inseriti nel filtro.
	Se il filtro non esiste, il comando restituisce $0$.

	\item \textbf{BFDEBUG}: Restituisce informazioni utili al debugging dell'implementazione del
	filtro. Questo comando accetta come primo parametro un sotto-comando; al momento, ne sono stati
	implementati due:

		\begin{itemize}
		\item \textbf{BFDEBUG STATUS <key>}: Restituisce una stringa che contiene informazioni sullo
		stato attuale del filtro, e in particolare il numero di filtri presenti nella catena, e
		l'errore che è stato specificato.

		\item \textbf{BFDEBUG FILTER <key> <idx>}: Restituisce informazioni su uno specifico filtro
		della catena; il filtro che si vuole ispezionare deve essere specificato come indice 
		nel parametro \verb|idx|. La stringa restituita mostra il numero di sezioni (o funzioni
		di hash), la dimensione in bit di una sezione, e il numero di bit impostati a 1.
		\end{itemize}

\end{itemize}

Si noti come, avendo esposto il solo parametro di probabilità, di fatto i comandi si applicano ad
una qualunque struttura dati probabilistica che implementa un insieme, consentendo in futuro di
cambiare implementazione se lo si ritenesse necessario, senza invalidare l'utilizzo da parte dei
client. I comandi di Redis, infatti, mantengono una retrocompatibilità completa fin dalla prima
versione, in modo che sia sempre possibile aggiornare il server senza causare problemi ai client.

\section{Esempio d'uso dei comandi}

Vediamo ora un esempio di sessione utilizzando i comandi appena descritti:

\medskip
\begin{lstlisting}[numbers=left,frame=single,caption=Esempio di utilizzo dei nuovi comandi per i filtri di Bloom]
127.0.0.1:6379> BFADD prova ELEMENTS a b c d e |\label{line:bfadd}|
(integer) 5
127.0.0.1:6379> BFCOUNT prova |\label{line:bfcount}|
(integer) 5
127.0.0.1:6379> BFADD prova ELEMENTS f g a |\label{line:bfadd2}|
(integer) 2
127.0.0.1:6379> BFCOUNT prova
(integer) 7
127.0.0.1:6379> BFEXIST prova b |\label{line:bfexist}|
(integer) 1
127.0.0.1:6379> BFEXIST prova z
(integer) 0
127.0.0.1:6379> BFDEBUG STATUS prova |\label{line:bfdebugstatus}|
"n:1 e:0.003"
127.0.0.1:6379> BFDEBUG FILTER prova 0 |\label{line:bfdebugfilter}|
"k:11 s:1798 b:77"
\end{lstlisting}

Per cominciare, alla linea~\ref{line:bfadd} viene creato un filtro chiamato \verb|prova| in cui
vengono inseriti 5 elementi (ciascun elemento è una stringa di un solo carattere, corrispondenti
alle prime cinque lettere dell'alfabeto). Il comando restituisce il valore $5$, corrispondente al
numero di elementi inseriti. Alla linea~\ref{line:bfcount}, viene effettuato un calcolo della 
cardinalità stimata, che restituisce il valore corretto $5$.

Alla linea~\ref{line:bfadd2} viene fatto un secondo inserimento; questa volta, oltre a due
elementi nuovi (stringhe \verb|f| e \verb|g|), viene inserito anche un elemento che era stato
già precedentemente inserito: la lettera \verb|a|. Il comando restituisce $2$, riportando
correttamente il numero di nuovi elementi inseriti, e anche il successivo calcolo della
cardinalità è corretto, poiché restituisce $7$

Alla linea~\ref{line:bfexist} viene effettuato un primo test di appartenenza al filtro tramite
il comando \verb|BFEXIST|, richiedendo
se l'elemento corrispondente alla stringa \verb|b| è presente; l'elemento era stato inserito
precedentemente (linea~\ref{line:bfadd}), ed infatti il comando restituisce correttamente $1$ per
indicare che l'elemento è presente. Successivamente, viene effettuato un secondo test con
l'elemento \verb|z| che non era stato mai inserito, e il comando questa volta restituisce $0$.

Alla linea~\ref{line:bfdebugstatus} viene usato il comando \verb|BFDEBUG STATUS| per ispezionare lo
stato del filtro. La stringa restituita è composta da due valori:

\medskip
\begin{tabular}{ |l|l|p{280px}| }
  \hline
  $n$ & $1$ & La catena del filtro scalabile contiene un solo filtro \\
  $e$ & $0.003$ & La probabilità di falsi positivi è impostata a \SI{0.3}{\percent} \\
  \hline
\end{tabular}
\medskip

Alla linea ~\ref{line:bfdebugfilter}, viene eseguito il comando \verb|BFDEBUG FILTER| per
ispezionare lo stato del primo filtro della catena. La stringa restituita contiene tre valori:

\medskip
\begin{tabular}{ |l|l|p{280px}| }
  \hline
  $k$ & $11$ & Il filtro utilizza $11$ funzioni di hash, cioè è diviso in $11$ sezioni \\
  $s$ & $1798$ & Ciascuna sezione è formata da \SI{1798}{\bit}, per un totale di $\num{1798 x 11} =
  \SI{19778}{\bit}$ \\
  $b$ & $77$ & Al momento ci sono \SI{77}{\bit} impostati a $1$. Questo corrisponde infatti a $7$
  elementi inseriti con $11$ funzioni di hash, e senza che si sia verificata alcuna collisione. \\
  \hline
\end{tabular}
\medskip


% ********************************************************************
% Backmatter
%*******************************************************
\appendix
\cleardoublepage
\chapter{Codice sorgente}

\section{Esempio di stored procedure in Redis}

Il seguente listato mostra una stored procedure scritta in LUA che dato un insieme ordinato
\verb|paying-users| in cui le stringhe sono gli ID degli utenti di un sistema, e il punteggio
associato è il totale speso dall'utente nell'applicazione, calcola qual è la media di spesa degli
utenti ``premium'', cioè che hanno speso più di €1000.

Si veda \ref{redislua} per maggiori informazioni.

\lstinputlisting[language={[5.0]Lua}]{code/zsetaverage.lua}


\section{Calcolo dell'errore di cardinalit\'a di un filtro di Bloom}

Il seguente listato effettua una simulazione di calcolo dell'errore di cardinalità in un filtro
di Bloom di 8192 bit, con 2 funzioni di hash.

Si veda \ref{sec:bloomcard} per maggiori informazioni.

\lstinputlisting[language=Python]{code/cardinality.py}


%********************************************************************
% Other Stuff in the Back
%*******************************************************
\cleardoublepage\include{FrontBackmatter/Bibliography}
\cleardoublepage\include{FrontBackmatter/Declaration}
\cleardoublepage\pagestyle{empty}

\hfill

\vfill


\pdfbookmark[0]{Colophon}{colophon}
\section*{Colophon}

Questo documento è stato rilasciato con licenza Creative Commons Attribuzione
3.0 Unported (CC BY 3.0). Per leggere il testo integrale della licenza si può
visitare il sito web \url{http://creativecommons.org/licenses/by/3.0/legalcode}. 
Il testo è anche disponibile in italiano, in formato ridotto, all'indirizzo web
\url{http://creativecommons.org/licenses/by/3.0/deed.it}.

La tesi è stata scritta in \LaTeX{} sul sistema operativo Apple macOS, con
l'editor di testo Sublime Text. Lo stile tipografico utilizzato è \texttt{classicthesis},
sviluppato da Andr\'e Miede and Ivo Pletikosić, ed è disponibile all'indirizzo
\url{https://bitbucket.org/amiede/classicthesis/}.

È possibile scaricare l'intero codice sorgente di questo documento, comprensivo 
di tutto il codice sorgente, dal repositorio GitHub:
\begin{center}
\url{https://github.com/rasky/thesis}
\end{center}

Le modifiche a Redis, oggetto di questa tesi, sono invece disponibili sul mio 
fork su Github:
\begin{center}
\url{https://github.com/rasky/redis}
\end{center}

\bigskip

\noindent\finalVersionString

%Hermann Zapf's \emph{Palatino} and \emph{Euler} type faces (Type~1 PostScript fonts \emph{URW
%Palladio L} and \emph{FPL}) are used. The ``typewriter'' text is typeset in \emph{Bera Mono},
%originally developed by Bitstream, Inc. as ``Bitstream Vera''. (Type~1 PostScript fonts were made
%available by Malte Rosenau and
%Ulrich Dirr.)

%\paragraph{note:} The custom size of the textblock was calculated
%using the directions given by Mr. Bringhurst (pages 26--29 and
%175/176). 10~pt Palatino needs  133.21~pt for the string
%``abcdefghijklmnopqrstuvwxyz''. This yields a good line length between
%24--26~pc (288--312~pt). Using a ``\emph{double square textblock}''
%with a 1:2 ratio this results in a textblock of 312:624~pt (which
%includes the headline in this design). A good alternative would be the
%``\emph{golden section textblock}'' with a ratio of 1:1.62, here
%312:505.44~pt. For comparison, \texttt{DIV9} of the \texttt{typearea}
%package results in a line length of 389~pt (32.4~pc), which is by far
%too long. However, this information will only be of interest for
%hardcore pseudo-typographers like me.%
%
%To make your own calculations, use the following commands and look up
%the corresponding lengths in the book:
%\begin{verbatim}
%    \settowidth{\abcd}{abcdefghijklmnopqrstuvwxyz}
%    \the\abcd\ % prints the value of the length
%\end{verbatim}
%Please see the file \texttt{classicthesis.sty} for some precalculated
%values for Palatino and Minion.
%
%    \settowidth{\abcd}{abcdefghijklmnopqrstuvwxyz}
%    \the\abcd\ % prints the value of the length

% ********************************************************************
% Game Over: Restore, Restart, or Quit?
%*******************************************************
\end{document}
% ********************************************************************
