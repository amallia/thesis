%*******************************************************
% Acknowledgments
%*******************************************************
\pdfbookmark[1]{Ringraziamenti}{ringraziamenti}

\begin{flushright}
{
	\emph{
		Ex quo exsistit et illud multa esse probabilia, quae, \\
		quamquam non perciperentur, tamen, \\
		quia visum quendam haberent insignem et inlustrem, \\
		his sapientis vita regeretur. \\
	}

	\medskip

	Ne deriva che vi sono delle conoscenze probabili le quali, \\
	benché non possano essere compiutamente accertate, \\
	appaiono così nobili ed elevate \\
	da poter fungere da guida per il saggio. \\

	\medskip

	\textbf{--- Cicerone, De Natura Deorum, I, 12} 
}
\end{flushright}

\bigskip

\begingroup
\let\clearpage\relax
\let\cleardoublepage\relax
\let\cleardoublepage\relax
\chapter*{Ringraziamenti}

Arrivando al termine del percorso a quasi venti anni dall'inizio, è
impossibile per me elencare tutte le persone che, di periodo in periodo, mi
hanno sostenuto. Rimangono però incisi nella mia memoria dei ricordi che
voglio riportare: Fabio Bardella con il quale ho studiato per i primissimi
esami; tutti i ragazzi del laboratorio di Via Cesalpino (Roccio, Mandingo,
Cinfa e Lollo in primis) con i quali ho affrontato tanti corsi; poi Andrea
Grandi con il quale ho fatto l'ultimo sprint, arrivando ad un passo dal
traguardo; ed infine Diego Piacentini, che mi ha dato l'ultimo proverbiale
colpo di grazia.

\bigskip

Ma senza dimenticare i miei cari: i miei fratelli che portano sempre gioia
nella mia vita, mia mamma e mio babbo che mi hanno da sempre incitato a
continuare senza mai dubitare che sarei arrivato in fondo; mia nonna che
testardamente ha continuato per vent'anni a chiedermi quando mi sarei
laureato, senza dimenticarsi di farlo neppure una volta; e mia moglie
Alessandra, che mi ha sopportato e amato durante buona parte di questi
vent'anni, aiutandomi nei momenti di difficoltà, incoraggiandomi e standomi
vicina, anche quando le ho potuto dedicare meno tempo di quello che avrei voluto.
\endgroup
